\newpage
\lecture{21}{06.07.2021}{}
We will use the following classification rules:
\begin{mdframed}[frametitle=Classification rules for $\mathcal{EL}$, nobreak=true]
	\begin{align*}
		&\infer[\text{CR1}]{A \sqsubseteq A}\\
		&\infer[\text{CR2}]{A \sqsubseteq \top}\\\\\\
		&\infer[\text{CR3}]{A_1 \sqsubseteq A_3}{A_1 \sqsubseteq A_2 & A_2 \sqsubseteq A_3}\\\\
		&\infer[\text{CR4}]{A \sqsubseteq B}{A \sqsubseteq A_1 & A \sqsubseteq A_2 & A_1 \sqcap A_2 \sqsubseteq B}\\\\
		&\infer[\text{CR5}]{A \sqsubseteq B}{A \sqsubseteq \exists r.A_1 & A_1 \sqsubseteq B_1 & \exists r.B_1 \sqsubseteq B}
	\end{align*}
\end{mdframed}
\begin{note}
	The above rules are, of course, not concrete rules, but rule schemata.
	We only allow instantiations of these rule schemata
	(i.e.\ replacements of the meta-variables $A, A_1, A_2, B, B_1$ by concrete $\mathcal{EL}$ concepts and $r$ by a concrete role name),
	in which all the GCIs occurring in the rule are $\mathcal{T}$-sequents.

	An application of the rule is possible if all the $\mathcal{T}$-sequents above the line occur in the current TBox $\mathcal{T}'$,
	and adds the $\mathcal{T}$-sequent below the line to $\mathcal{T}'$ unless it already belongs to $\mathcal{T}'$.
\end{note}

We saturate a TBox $\mathcal{T}$ by exhaustively applying the classification rules to it.
The resulting TBox $\mathcal{T}^{*}$ is called the \textit{saturated} TBox.

\stepcounter{definition}
\begin{lemma}
	The saturated TBox $\mathcal{T}^*$ is uniquely determined by $\mathcal{T}$,
	and it can be computed by a polynomial number of rule applications.
\end{lemma}
\begin{proof}
	Each rule application adds one new $\mathcal{T}$-sequent to the current TBox $\mathcal{T}'$.
	Thus, after a polynomial number of rule applications, no new $\mathcal{T}$-sequent can be added,
	and the saturation terminates.

	The choice of which applicable rule to apply during saturation does not influence the resulting TBox.
	Note that we only add $\mathcal{T}$-sequents, but never remove anything.
	Thus, if the condition of a rule is satisfied, it stays satisfied.
	Thus, the consequence of the rule must be added at some point during saturation,
	either by this rule or another one.
\end{proof}

To show that polynomial-time saturation of $\mathcal{EL}$ TBoxes yields a polynomial-time classification procedure,
it is sufficient to prove the following equivalence:
\[
	\mathcal{T} \vDash A \sqsubseteq B \iff A \sqsubseteq B \in \mathcal{T}^*
\]
for all concept names $A, B$ in $\mathcal{T}$.
Proving this is equivalence is proving soundness and completeness of the classification procedure.

\begin{lemma}[Soundness]
	If all the GCIs in $\mathcal{T}'$ follow from $\mathcal{T}$
	and the $\mathcal{T}$-sequents above the line of one of the rules belong to $\mathcal{T}'$,
	then the $\mathcal{T}$-sequent below the line also follows from $\mathcal{T}$.
\end{lemma}
\begin{proof}
	This is an immediate consequence of the following facts:
	\begin{itemize}
		\item the subsumption relation is reflexive and transitive
		\item $\top$ subsumes any concept w.r.t.\ any TBox
		\item $A \sqsubseteq_{\mathcal{T}} A_1, A \sqsubseteq_{\mathcal{T}} A_2$ implies $A \sqsubseteq_{\mathcal{T}} A_1 \sqcap A_2$
		\item $A_1 \sqsubseteq_{\mathcal{T}} A_2 $ implies $\exists r.A_1 \sqsubseteq_{\mathcal{T}} \exists r.A_2$
			 \qedhere
	\end{itemize}
\end{proof}

For completeness, we prove its contrapositive:
\[
	A \sqsubseteq B \notin \mathcal{T}^* \implies \mathcal{T} \not\vDash A \sqsubseteq B
.\]
To do so, we will use $\mathcal{T}^*$ to construct a \textit{canonical model} of $\mathcal{T}$,
that does not satisfy $A \sqsubseteq B$ in the case of $A \sqsubseteq B \in \mathcal{T}^*$:
\begin{definition}[Canonical model]
	Let $\mathcal{T}$ be a general $\mathcal{EL}$ TBox in normal form and $\mathcal{T}^*$ the saturated TBox
	obtained by exhaustive application of the classification rules.
	
	The canonical interpretation $\mathcal{I}_{\mathcal{T}^*}$ induced by $\mathcal{T}^*$ is defined as follows:
	\begin{itemize}
		\item $\Delta^{\mathcal{I}_{\mathcal{T}^*}} \coloneqq \left\{ A \mid A \text{ is a concept name in } \func{sig}(\mathcal{T}) \right\} \cup \left\{ \top \right\}$,
		\item $A^{\mathcal{I}_{\mathcal{T}^*}} \coloneqq \left\{ B \in \Delta^{\mathcal{I}_{\mathcal{T}^*}} \mid B \sqsubseteq A \in \mathcal{T}^* \right\}$ for all concept names $A \in \func{sig}(\mathcal{T})$,
		\item $r^{\mathcal{I}_{\mathcal{T}^*}} \coloneqq \left\{ (A,B) \in \Delta^{\mathcal{I}_{\mathcal{T}^*}} \times \Delta^{\mathcal{I}_{\mathcal{T}^*}} \mid A \sqsubseteq \exists r.B \in \mathcal{T}^* \right\}$ \\
			for all role names $r \in \func{sig}(\mathcal{T})$.
	\end{itemize}
\end{definition}
\begin{note}
	By definition, we have $B \in A^{\mathcal{I}_{\mathcal{T}^*}} \iff B \sqsubseteq A \in T^*$ for all concept names $A \in \func{sig}(\mathcal{T})$.
	And the same is true for $A = \top$.
\end{note}
\begin{lemma}
	The canonical interpretation induced by $\mathcal{T}^*$ is a model of the saturated TBox $\mathcal{T}^*$.
\end{lemma}
\begin{proof}
	All the GCIs in $\mathcal{T}^*$ are $\mathcal{T}$-sequents,
	i.e.\ the are of the form described in definition 6.7.
	Therefore, we distinguish the following cases:
	 \begin{itemize}
		\item $A \sqsubseteq B \in \mathcal{T}^*$:
			If $A' \in A^{\mathcal{I}_{\mathcal{T}^*}}$, then we have $A' \sqsubseteq A \in \mathcal{T}^*$.
			Since $\mathcal{T}^*$ is saturated, CR3 is not applicable,
			and thus $A' \sqsubseteq B \in \mathcal{T}^*$.
			Thus, $A' \in B^{\mathcal{I}_{\mathcal{T}^*}}$, which shows that $\mathcal{I}_{\mathcal{T}^*}$
			satisfies  $A \sqsubseteq B$.
		\item $A_1 \sqcap A_2 \sqsubseteq B \in \mathcal{T}^*$: can be treated similarly, using CR4.
		\item $A \sqsubseteq \exists r.B \in \mathcal{T}^*$:
			If  $A' \in A^{\mathcal{I}_{\mathcal{T}^*}}$, then $A' \sqsubseteq A \in \mathcal{T}^*$,
			and thus (due to CR3) $A' \sqsubseteq \exists r.B \in \mathcal{T}^*$.
			The definition of the interpretation of roles than yields $(A', B) \in r^{\mathcal{I}_{\mathcal{T}^*}}$.
			Finally, due to CR1, we have $B \sqsubseteq B \in \mathcal{T}^*$, and thus $B \in B^{\mathcal{I}_{\mathcal{T}^*}}$.
			This shows $A' \in (\exists r.B)^{\mathcal{I}_{\mathcal{T}^*}}$.
		\item $\exists r.A \sqsubseteq B \in \mathcal{T}^*$:
			If $A' \in (\exists r.A)^{\mathcal{I}_{\mathcal{T}^*}}$,
			then there is $B' \in \Delta^{\mathcal{I}_{\mathcal{T}^*}}$ such that $(A', B') \in r^{\mathcal{I}_{\mathcal{T}^*}}$ 
			and $B' \in A^{\mathcal{I}_{\mathcal{T}^*}}$.
			This yields $A' \sqsubseteq \exists r.B' \in \mathcal{T}^*$ 
			and $B' \sqsubseteq A \in \mathcal{T}^*$.
			By CR5, we also have $A' \sqsubseteq B \in \mathcal{T}^*$,
			which yields $A' \in B^{\mathcal{I}_{\mathcal{T}^*}}$.
			\qedhere
	\end{itemize}
\end{proof}

\begin{lemma}[Completeness]
	Let $\mathcal{T}$ be a general $\mathcal{EL}$ TBox in normal form
	and $\mathcal{T}^*$ the saturated TBox obtained by exhaustive application of the classification rules.
	Then
	\[
	\mathcal{T} \vDash A \sqsubseteq B \implies A \sqsubseteq B \in \mathcal{T}^*
	.\]
\end{lemma}
\begin{proof}
	We show the contrapositive.
	Assume that $A \sqsubseteq B \notin \mathcal{T}^*$.
	Then $A \notin B^{\mathcal{I}_{\mathcal{T}^*}}$.
	By CR1, we have $A \sqsubseteq A \in \mathcal{T}^*$, and thus $A \in A^{\mathcal{I}_{\mathcal{T}^*}}$.
	Since $\mathcal{I}_{\mathcal{T}^*}$ is a model of  $\mathcal{T}^*$,
	it is also a model of its subset $\mathcal{T} \subseteq \mathcal{T}^*$.
	This shows that $\mathcal{T} \not\vDash A \sqsubseteq B$.
\end{proof}

\begin{theorem}
		Subsumption in $\mathcal{EL}$ w.r.t.\ general TBoxes is decidable in polynomial time.
\end{theorem}

\newpage
\section{Subsumption in $\mathcal{ELI}$ w.r.t.\ general TBoxes}
We now take a look at subsumption in $\mathcal{ELI}$.
We will see that, in contrast to $\mathcal{EL}$, subsumption in $\mathcal{ELI}$ w.r.t.\ general TBoxes
is no longer polynomial, but \textsc{ExpTime}-complete.
This is caused by the ability of $\mathcal{ELI}$ to express a restricted form of value restrictions:
\[
\exists r^-.C \sqsubseteq D \text{ has the same models as } C \sqsubseteq \forall r.D
.\]

\subsection{Auxiliary results}
We now show the \textsc{ExpTime} upper bound.
Again, we can assume a general $\mathcal{ELI}$ TBox $\mathcal{T}$ to be in i.normal form (or i.normalized)
if all its GCIs are of one of the following forms:
\begin{itemize}
	\item $A \sqsubseteq B$
	\item $A_1 \sqcap A_2 \sqsubseteq B$
	\item $A \sqsubseteq \exists r.B$
	\item $A \sqsubseteq \forall r.B$
\end{itemize}
where $A, A_1, A_2, B$ are concept names or the top-concept $\top$
and $r$ is a role name or the inverse of a role name.

\begin{corollary}
	Given a general $\mathcal{ELI}$ TBox $\mathcal{T}$, we can compute in polynomial time
	an i.normalized $\mathcal{ELI}$ TBox $\mathcal{T}'$ that is a conservative extension of $\mathcal{T}$.
	
	In particular, we have
	\[
	\mathcal{T} \vDash A \sqsubseteq B \iff \mathcal{T}' \vDash A \sqsubseteq B
	\]
	for all concept names $A, B \in \func{sig}(\mathcal{T})$.
\end{corollary}

\subsection{Classification procedure for $\mathcal{ELI}$}
We assume that the input TBox $\mathcal{T}$ is a general $\mathcal{ELI}$ TBox in i.normal form.
We need an extended and more complex notion of sequents:
\begin{definition}
	A $\mathcal{T}$-i.sequent is an expression of the form
	\[
		K \sqsubseteq \left\{ A \right\}, K \sqsubseteq \exists r.K', \text{ or } K \sqsubseteq \forall r.\left\{ A \right\}
	,\]
	where $K, K'$ are sets of concept names in $\func{sig}(\mathcal{T})$,
	$A$ is a concept name in $\func{sig}(\mathcal{T})$,
	and $r $ is a role name in $\func{sig}(\mathcal{T})$ or the inverse of a role name in $\func{sig}(\mathcal{T})$.
\end{definition}
\begin{note}
	The overall number of $\mathcal{T}$-i.sequents is exponential in the size of $\mathcal{T}$.
	A set in a  $\mathcal{T}$-i.sequent stands for the conjunction of its elements.
	The empty conjunction is $\top$.
	$\mathcal{T}$-i.sequents are GCIs, and a set of $\mathcal{T}$-i.sequents is a general $\mathcal{ELI}$ TBox.
	Every GCI in the i.normalized TBox $\mathcal{T}$ is either equivalent to a $\mathcal{T}$-i.sequent
	or a tautology, i.e.\ satisfied in every interpretation.
\end{note}
