\lecture{6}{29.04.2021}{}

\begin{proof}
	By induction on the structure of $C$ :
	\begin{itemize}
		\item $C = A \in \mathscr{C}$:
			Then $d_1 \in A^{\mathcal{I}_1}$ iff $d_2 \in A^{\mathcal{I}_2}$ is an immediate consequence of $d_1 \rho d_2$ for a bisimulation $\rho$.
		\item $C = D \sqcap E$ :
			$d_1 \in \left( D \sqcap E \right)^{\mathcal{I}_1}$ iff $d_1 \in D^{\mathcal{I}_1}$ and $d_1 \in E^{\mathcal{I}_1}$.
			But then $d_2 \in D^{\mathcal{I}_2}$ and $d_2 \in E^{\mathcal{I}_2}$ (induction).
			And therefore $d_2 \in \left( D \sqcap E \right)^{\mathcal{I}_2}$.
		\item $\neg, \sqcup$ can be treated similarly.
		\item $C = \exists r.D$ :
				$d_1 \in \left( \exists r.D \right)^{\mathcal{I}_1}$ iff there is $d_1' \in \Delta^{\mathcal{I}_1}$ such that $(d_1, d_1') \in r^{\mathcal{I}_1}$ and $d_1' \in D^{\mathcal{I}_1}$.
				Then there is $d_2' \in \Delta^{\mathcal{I}_2}$ such that $\left( d_2, d_2' \right) \in r^{\mathcal{I}_2}$ and $d_1' \rho d_2'$.
				By induction we get that $d_2' \in D^{\mathcal{I}_2}$ and therefore  $d_2 \in \left( \exists r.D \right)^{\mathcal{I}_2}$.
			\item $C = \forall r.D$ can be treated similarly.
	\end{itemize}
\end{proof}

\section{Expressive power}
We have so far introduced extensions of $\mathcal{ALC}$ by concept constructors \textbf{number restrictions}, \textbf{nominals} and the role constructor \textbf{invers roles}.
We will now show, that they really extend $\mathcal{ALC}$, i.e., there is no way these could be expressed by plain $\mathcal{ALC}$.

\begin{prop}[$\mathcal{ALCN}$ is more expressive than $\mathcal{ALC}$]
	No $\mathcal{ALC}$-concept description is equivalent to the $\mathcal{ALCN}$-concept description $\left( \leq 1 r \right)$.
\end{prop}
\begin{proof}
	Assume that $C$ is an $\mathcal{ALC}$-concept description such that $C \equiv ( \leq 1 r )$, i.e.\ $C^{\mathcal{I}} = \left( \leq 1 r \right)^\mathcal{I}$ holds for all interpretations $\mathcal{I}$.
	Consider the following two interpretations $\mathcal{I}_1$ and $\mathcal{I}_2$ :
	\begin{figure}[H]
		\centering
		\begin{subfigure}[t]{.475\textwidth}
			\centering
			\begin{tikzpicture}
				\node[default] (d1) {$d_1$};
				\node[default, below of = d1] (e1) {$e_1$};
				\draw (d1) edge[left] node{$r$} (e1);
			\end{tikzpicture}
			\caption{$\mathcal{I}_1$}
		\end{subfigure}
		\hfill
		\begin{subfigure}[t]{.475\textwidth}
			\centering
			\begin{tikzpicture}
				\node[default] (d2) {$d_2$};
				\node[default, below left of = d2] (e21) {$e_{2,1}$};
				\node[default, below right of = d2] (e22) {$e_{2,2}$};
				\draw (d2) edge[left] node{$r$} (e21);
				\draw (d2) edge[right] node{$r$} (e22);
			\end{tikzpicture}
			\caption{$\mathcal{I}_2$}
		\end{subfigure}
	\end{figure}
	Then $\rho \coloneqq \left\{ (d_1,d_2), (e_1,e_{1,2}), (e_1,e_{2,2}) \right\}$ is a bisimulation, which shows $(\mathcal{I}_1, d_1) \sim (\mathcal{I}_2,d_2)$.
	Thus, we have $d_1 \in C^{\mathcal{I}_1} \iff d_2 \in C^{\mathcal{I}_2}$.
	This yields a contradiction, since $d_1 \in ( \leq 1r)^{\mathcal{I}_1} = C^{\mathcal{I}_1}$ but $d_2 \notin ( \leq 1r)^{\mathcal{I}_2} = C^{\mathcal{I}_2}$. $\contra$
\end{proof}

\begin{prop}[$\mathcal{ALCI}$ is more expressive than $\mathcal{ALC}$]
	No $\mathcal{ALC}$-concept description is equivalent to the $\mathcal{ALCI}$-concept description $\exists r^{-}.\top$.
\end{prop}
\begin{proof}
	Assume that $C$ is an $\mathcal{ALC}$-concept description such that $C \equiv \exists r^{-}.\top$.
	Consider the following two interpretations:
	\begin{figure}[H]
		\centering
		\begin{subfigure}[t]{.475\textwidth}
			\centering
			\begin{tikzpicture}
				\node[default] (e1) {$e_1$};
				\node[default, below of = e1] (d1) {$d_1$};
				\draw (e1) edge[left] node{$r$} (d1);
			\end{tikzpicture}
			\caption{$\mathcal{I}_1$}
		\end{subfigure}
		\hfill
		\begin{subfigure}[t]{.475\textwidth}
			\centering
			\begin{tikzpicture}
				\node[default] (d2) {$d_2$};
			\end{tikzpicture}
			\caption{$\mathcal{I}_2$}
		\end{subfigure}
	\end{figure}
	Then $\rho \coloneqq \left\{ (d_1,d_2) \right\}$ is a bisimulation, which shows $\left(  \mathcal{I}_1, d_1 \right) \sim \left( \mathcal{I}_2,d_2 \right)$.
	Thus, we have  $d_1 \in C^{\mathcal{I}_1} \iff d_2 \in C^{\mathcal{I}_2}$.
	This yields a contradiction, since $d_1 \in \left( \exists r^{-}.\top \right)^{\mathcal{I}_1} = C^{\mathcal{I}_1}$, but
	$d_2 \notin \left( \exists r^{-}.\top \right)^{\mathcal{I}_2} = C^{\mathcal{I}_2}$. $\contra$
\end{proof}

\begin{prop}[$\mathcal{ALCO}$ is more expressive than $\mathcal{ALC}$]
	No $\mathcal{ALC}$-concept description is equivalent to the $\mathcal{ALCO}$-concept description $\left\{ a \right\}$.
\end{prop}
\begin{proof}
	Assume that $C$ is an $\mathcal{ALC}$-concept description such that $C \equiv \left\{ a \right\}$.
	Consider the following two interpretations:
	\begin{figure}[H]
		\centering
		\begin{subfigure}[t]{.475\textwidth}
			\centering
			\begin{tikzpicture}
				\node[default, label =left: $\left\{ a \right\}^{\mathcal{I}_1}$] (d1) {$d_1$};
			\end{tikzpicture}
			\caption{$\mathcal{I}_1$}
		\end{subfigure}
		\hfill
		\begin{subfigure}[t]{.475\textwidth}
			\centering
			\begin{tikzpicture}
				\node[default] (d2) {$d_2$};
				\node[default, label= right: $\left\{ a \right\}^{\mathcal{I}_2}$, right of = d2] (e2) {$e_2$};
			\end{tikzpicture}
			\caption{$\mathcal{I}_2$}
	\end{subfigure}
	\end{figure}
	The meaning of these interpretations is, that $a^{\mathcal{I}_1} = d_1$ and $a^{\mathcal{I}_2} = e_2$.
	Then $\rho \coloneqq \left\{ (d_1,d_2) \right\}$ is a bisimulation, which yields $\left( \mathcal{I}_1, d_1 \right) \sim \left( \mathcal{I}_2, d_2 \right)$.
	Thus, $d_1 \in C^{\mathcal{I}_1} \iff d_2 \in C^{\mathcal{I}_2}$.
	This yields a contradiction, since $d_1 \in \left\{ a \right\}^{\mathcal{I}_1} = C^{\mathcal{I}_1}$, but
	 $d_2 \notin \left\{ a \right\}^{\mathcal{I}_2} = C^{\mathcal{I}_2}$. $\contra$
\end{proof}
These proofs were the result of the fact, that we defined bisimulation with $\mathcal{ALC}$ in mind.
One could also define a notion of bisimulation under which $\mathcal{ALCN}$ or $\mathcal{ALCI}$ are invariant.

\section{Closure under disjoint union}
\begin{definition}
	Let $\mathfrak{R}$ be an index set and $\left( \mathcal{I}_{\nu} \right)_{\nu \in \mathfrak{R}}$ a family of interpretations $\mathcal{I}_{\nu} = \left( \Delta^{\mathcal{I}_{\nu}}, \cdot^{\mathcal{I}_{\nu}} \right)$.
	Their disjoint union $\mathcal{J}$ is defined as follows:
	\begin{align*}
		\Delta^{\mathcal{J}} &= \left\{ (d,\nu) \mid \nu \in \mathfrak{R}\land d \in \Delta^{\mathcal{I}_{\nu}} \right\}, \\
		A^{\mathcal{J}} &= \left\{ (d,\nu) \mid \nu \in \mathfrak{R} \land d \in A^{\mathcal{I}_\nu}\right\} \text{for all } A \in \mathscr{C}, \\
		r^{\mathcal{J}} &= \left\{ \left( (d,\nu),(e,\nu) \right) \mid \nu \in \mathfrak{R} \land (d,e) \in r^{\mathcal{I}_\nu} \right\} \text{for all } r \in \mathscr{R}.
	\end{align*}
\end{definition}
\begin{notation}
	We then write
	\[
	\mathcal{J} = \biguplus_{\nu \in \mathfrak{R}} \mathcal{I}_{\nu}
	.\]
\end{notation}
\begin{note}
	If there are some $\mathcal{I}_{\nu_1}, \mathcal{I}_{\nu_2}$ that are not disjoint,
	we make them artificially disjoint by adding the index $\nu$ to the tuple.
\end{note}

\begin{lemma}\label{lemma:disjoint union 1}
	For $\nu \in \mathfrak{R}$, all $\mathcal{ALC}$-concept description $C$, and all $d \in \Delta^{\mathcal{I}_\nu}$ we have:
	\[
		d \in C^{\mathcal{I}_\nu} \iff (d,\nu) \in C^{\mathcal{J}}
	.\]
\end{lemma}
\begin{proof}[Proof sketch]
	Let $\mathcal{J} = \biguplus_{\nu \in \mathfrak{R}} \mathcal{I}_\nu$.
	We claim that
	\[
	\rho_\nu \coloneqq \left\{ \left( (d,\nu),d \right) \mid d \in \Delta^{\mathcal{I}_\nu} \right\}
	\]
	is a bisimulation (and proof that).
	And then lemma \ref{lemma:disjoint union 1} follows from theorem \ref{thm: bisimulation invariance}.
\end{proof}

\begin{theorem}
	Let $\mathcal{T}$ be an $\mathcal{ALC}$ TBox and $(\mathcal{I}_\nu)_{\nu \in \mathfrak{R}}$ a family of models of $\mathcal{T}$.
	Then its disjoint union $\mathcal{J} = \biguplus_{\nu \in \mathfrak{R}} \mathcal{I}_\nu$ is also a model of $\mathcal{T}$.
\end{theorem}
This theorem tells us, that models of TBoxes are closed under disjoint union.
\begin{proof}
	Let $C \sqsubseteq D \in \mathcal{T}$.
	We must show that $C^{\mathcal{J}} \subseteq D^{\mathcal{J}}$.
	Let $\left( d,\nu \right) \in C^{\mathcal{J}}$.
	By lemma \ref{lemma:disjoint union 1}, this implies $d \in C^{\mathcal{I}_\nu}$.
	Since $\mathcal{I}_\nu$ is a model of $\mathcal{T}$, we obtain $d \in D^{\mathcal{I}_\nu}$.
	Again, by lemma \ref{lemma:disjoint union 1}, this implies $(d,\nu) \in D^{\mathcal{J}}$.
\end{proof}

\begin{corollary}
 	Let $\mathcal{T}$ be an $\mathcal{ALC}$ TBox and $C$ an $\mathcal{ALC}$ concept that is satisfiable w.r.t.\ $\mathcal{T}$.
	Then there is a model $\mathcal{J}$ of $\mathcal{T}$ in which the extension $C^{\mathcal{J}}$ of $C$ is infinite.
\end{corollary}
\begin{proof}
	Since $C$ is satisfiable w.r.t.\ $\mathcal{T}$, there is a model $\mathcal{I}$ of $\mathcal{T}$ and $d \in \Delta^{\mathcal{I}}$ such that $d \in C^{\mathcal{I}}$.
	For $\nu = 1,2,3,\ldots$ let $\mathcal{I}_\nu = \mathcal{I}$.
	Let $\mathcal{J} = \biguplus_{\nu \geq 1} \mathcal{I}_\nu$.
	We know that $\mathcal{J}$ is a model of $\mathcal{T}$ since $\mathcal{I}$ is a model of $\mathcal{T}$.
	In $\mathcal{J}$, $C$ has infinitely many elements, since $(d,\nu) \in C^{\mathcal{J}}$ for $\nu \geq 1$.
\end{proof}
Therefore the only concepts that are \textbf{always} interpreted as a finite set,
are unsatisfiable concepts (w.r.t.\ a TBox $\mathcal{T}$).

\section{Finite model property}
\begin{definition}[Finite model]
	The interpretation $\mathcal{I}$ is a model of a concept $C$ w.r.t.\ a TBox $\mathcal{T}$,
	if $\mathcal{I}$ is a model of $\mathcal{T}$ such that $C^{\mathcal{I}} \neq \emptyset$.
	We call this model finite if $\Delta^{\mathcal{I}}$ is finite.
	\newline
	We call the following the \textit{Finite model property} (of $\mathcal{ALC}$):
	\newline
	If $\mathcal{T}$ is an $\mathcal{ALC}$-TBox and $C$ an $\mathcal{ALC}$-concept description 
	such that $C$ is satisfiable w.r.t.\ $\mathcal{T}$,
	then $C$ has a finite model w.r.t.\ $\mathcal{T}$.
\end{definition}	
To proof this we first need some auxiliary results.

\begin{mdframed}[frametitle={Sizes in $\mathcal{ALC}$}]
	We define the size of an $\mathcal{ALC}$-concept description recursively:
	\begin{itemize}
		\item $C = A \in \mathscr{C}$ : $\text{size}(C) \coloneqq 1$,
		\item $C = C_1 \sqcap C_2$ or $C = C_1 \sqcup C_2$ : $\text{size}(C) \coloneqq 1 + \text{size}(C_1) + \text{size}(C_2)$,
		\item $C = \neg D$ or $C = \exists r.D$ or $C = \forall r.D$ : $\text{size}(C) \coloneqq 1 + \text{size}(D)$.
	\end{itemize}
	We also define the size of a TBox $\mathcal{T}$ as follows:
	\[
		\text{size}(\mathcal{T}) \coloneqq \sum_{C \sqsubseteq D \in \mathcal{T}} \text{size}(C) + \text{size}(D)
	.\]
\end{mdframed}

\begin{mdframed}[frametitle={Subconcepts in $\mathcal{ALC}$}]
	We define the subconcepts of an $\mathcal{ALC}$-concept description recursively:
	\begin{itemize}
		\item $C = A \in \mathscr{C}$ : $\text{sub}(C) \coloneqq \left\{ A \right\}$,
		\item $C = C_1 \sqcap C_2$ or $C = C_1 \sqcup C_2$ : $\text{sub}(C) \coloneqq \left\{ C \right\} \cup \text{sub}(C_1) \cup \text{sub}(C_2)$,
		\item $C = \neg D$ or $C = \exists r.D$ or $C = \forall r.D$ : $\text{sub}(C) \coloneqq \left\{ C \right\} \cup \text{sub}(D)$.
	\end{itemize}
	We also define the subconcepts of a TBox $\mathcal{T}$ as follows:
	\[
		\text{sub}(\mathcal{T}) \coloneqq \bigcup_{C \sqsubseteq D \in \mathcal{T}} \text{sub}(C) \cup \text{sub}(D)
	.\]
\end{mdframed}
