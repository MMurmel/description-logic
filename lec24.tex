\newpage
\lecture{24}{15.07.2021}{}
We will focus on the complexity of conjunctive query answering and consider the query entailment decision problem:
\begin{definition}[query entailment]
	Let $q$ be a conjunctive query of arity $k$, $\mathcal{I}$ an interpretation
	and $\vec{a} = a_1, \ldots, a_k$ a tuple of individuals.

	We say that $\mathcal{I}$ \textit{entails} $q(\vec{a})$ (and write $\mathcal{I} \vDash q(\vec{a})$) if $\vec{a} \in \func{ans}(q, \mathcal{I})$.
	If $k = 0$, then we call $q$ a Boolean query and simply write $\mathcal{I} \vDash q$.
\end{definition}

\begin{prop}
	The query entailment problem for conjunctive queries in \textsc{NP}-complete.
\end{prop}
\begin{proof}
	\underline{In \textsc{NP}:}
	guess a mapping $\pi: \func{var}(q) \to \Delta^{\mathcal{I}}$ and test whether it is an $\vec{a}$-match.
	
	\underline{\textsc{NP}-hard:}
	by a reduction from 3-colorability.
	The (undirected) graph $G = (V,E)$ is 3-colorable if there exists a mapping
	$c: V \to \left\{ \var{red}, \var{blue}, \var{green} \right\}$ 
	such that $\left\{ u,v \right\} \in E$ implies $c(u) \neq c(v)$.
	For a given graph $G = (V, E)$ with $\lvert V \rvert = k$ we construct $q_G$ as follows:
	\[
		q_G = \exists v_1, \ldots, \exists v_k. \bigwedge_{\left\{ u,v \right\} \in E} E(u,v) 
	,\]
	and use interpretation $\mathcal{I}$ with $\Delta^{\mathcal{I}} = \left\{ \var{red}, \var{blue}, \var{green} \right\}$.
	It is then easy to see that
	\[
	\mathcal{I} \vDash q \iff G \text{ is 3-colorable} \qedhere
	\]
\end{proof}

The fact that query entailment is \textsc{NP}-complete does not contradict the fact that,
in practice, there are highly efficient relational DB engines that scale very well even for huge databases.
This is true, because the size of the data in DB systems is usually large, whereas the size of queries is small.
But in the reduction the query had the size of the graph, and the data had constant size.

In order to take this mismatch into account, in the database setting, one often considers data complexity,
instead of the usual (combined) complexity.
\begin{mdframed}[frametitle=Data complexity]
	Measure the complexity in the size of the data only,
	and assume that the query has constant size.
\end{mdframed}

\begin{prop}
	The query entailment problem for conjunctive queries is in \textsc{P} w.r.t.\ data complexity.
\end{prop}
\begin{proof}
	Generate all mappings $\pi: \func{var}(q) \to \Delta^{\mathcal{I}}$ 
	and test whether any of them is an $\vec{a}$-match.
	There are $\lvert \Delta^\mathcal{I} \rvert^{\lvert \func{var}(q) \rvert}$ such mappings.
	Since the query is considered constant, these are polynomially many.
\end{proof}
One can even show that the query entailment problem for FO queries (and thus also conjunctive queries)
belongs to a complexity class strictly contained in \textsc{P} w.r.t.\ data complexity.

\begin{theorem}
	The query entailments problem for FO queries is in $\textsc{AC}^{0}$ w.r.t.\ data complexity.
\end{theorem}
\begin{note}
	$\textsc{AC}^0 \subset \textsc{LogSpace} \subseteq \textsc{P}$
\end{note}

\section{Ontology-mediated query answering}
In ontology-mediated query answering (OMQA) we consider:
\begin{itemize}
	\item a TBox $\mathcal{T}$ that represents background knowledge,
	\item an ABox $\mathcal{A}$ that gives an incomplete description of the data,
	\item a conjunctive query $q$.
\end{itemize}
The actual date (i.e.\ the interpretation $\mathcal{I}$) is not (yet) known,
but it must be consistent with $\mathcal{K} = (\mathcal{T},\mathcal{A})$.
We want to compute the answers of $q$ that hold for all possible data,
i.e.\ for all models of $\mathcal{K}$.
These answers are called \textit{certain answers}.

\begin{definition}[certain answers]
	Let $\mathcal{K} = (\mathcal{T}, \mathcal{A})$ be a knowledge base.
	Then $\vec{a}$ is a certain answer to $q$ on $\mathcal{K}$ if
	\begin{itemize}
		\item all individual names from $\vec{a}$ occur in $\mathcal{A}$ and
		\item $\vec{a} \in \func{ans}(q,\mathcal{I})$ for every model $\mathcal{I}$ of $\mathcal{K}$.
	\end{itemize}

	We use $\func{cert}(q, \mathcal{K})$ to denot the set of all certain answers to $q$ on $\mathcal{K}$ i.e.\
	\[
		\func{cert}(q, \mathcal{K}) = \bigcap_{\mathcal{I} \text{ model of } \mathcal{K}} \func{ans}(q,\mathcal{I})
	.\]
\end{definition}
\begin{note}
	$\vec{a} \in \func{cert}(q,\mathcal{K}) \iff \mathcal{K} \vDash q(\vec{a})$
\end{note}

In the context of OMQA, we consider the following query entailment problem:
\begin{definition}[OMQA query entailment]
	Let $q$ be a conjunctive query of arity $k$, $\mathcal{K}$ a knowledge base
	and $\vec{a} = a_1, \ldots, a_k$ a tuple of individuals occurring in $\mathcal{K}$.

	We say that $\mathcal{K}$ \textit{entails} $q(\vec{a})$ (and write $\mathcal{K} \vDash q(\vec{a})$)
	if $\vec{a} \in \func{cert}(q, \mathcal{K})$.
	If $k=0$, then we simply write $\mathcal{K} \vDash q$.
\end{definition}

\begin{mdframed}[frametitle= Data complexity in OMQA]
	Consider only \textit{simple} ABoxes, whose assertions are of the form
	$a:A$ and $(a,b):r$ where $A \in \mathscr{C}$ and $r \in \mathscr{R}$.

	Measure the complexity in the size of the ABox only,
	and assume that the TBox and the query have constant size.
\end{mdframed}

The complexity of OMQA query entailment of course depends on which query language
and which DL for formulating the KB are used.

\underline{Query language:}
\newline
We will only consider conjunctive queries,
because for FO queries, OMQA query entailment would be undecidable
(Because it is equivalent to reasoning in FOL).

\underline{Description Logic:}
\begin{itemize}
	\item $\mathcal{ALC}$: \textsc{coNP}-complete
	\item $\mathcal{EL}$: \textsc{P}-complete
	\item DL-Lite: $\textsc{AC}^0$
\end{itemize}
In the following, we will only focus on the corresponding hardness results.

\subsection{data complexity of OMQA in $\mathcal{ALC}$}
\medskip
\begin{prop}
	In $\mathcal{ALC}$, the query entailment problem of conjunctive queries is \textsc{coNP}-hard w.r.t.\ data complexity.
\end{prop}
\begin{proof}
	By reduction of non-3-colorability.
	Because of data complexity, the resulting TBox and the query will not depend on the input graph, i.e.\ be constant:
	\begin{alignat*}{2}
	\mathcal{T} &=\big\{&\top &\sqsubseteq R \sqcup G \sqcup B \big.\\
				&&R \sqcap \exists r.R &\sqsubseteq D\\
				&&G \sqcap \exists r.G &\sqsubseteq D\\
				&&\big. B \sqcap \exists r.B &\sqsubseteq D \big\}\\
		q &= \exists x. D(x)
	\end{alignat*}

	The input graph $G=(V,E)$ is translated into the ABox
	\[
		\mathcal{A}_G \coloneqq \left\{ (u,v):r \mid \left\{u,v\right\} \in E\right\}
	.\]
	We have $(\mathcal{T}, \mathcal{A}_G) \vDash q \iff G$ is not 3-colorable.
	
	"$\implies$": By its contrapositive.
	Assume that $G$ is 3-colorable and let $c: V \to \left\{ \var{red}, \var{blue}, \var{green} \right\}$ 
	be the coloring.
	Let $\mathcal{I}_c$ be the interpretation with
	\begin{itemize}
		\item $\Delta^{\mathcal{I}_c} \coloneqq V$
		\item $v^{\mathcal{I}_c} = v$ 
		\item $r^{\mathcal{I}_c} \coloneqq \left\{ (u,v) \mid \left\{ u, v \right\} \in E \right\}$
	\end{itemize}
	where $v$ is an individual name.
	For the concepts we have:
	\begin{itemize}
		\item $R^{\mathcal{I}_c} \coloneqq \left\{ v \in V \mid c(v) = \var{red}\right\}$
		\item $B^{\mathcal{I}_c} \coloneqq \left\{ v \in V \mid c(v) = \var{blue}\right\}$
		\item $G^{\mathcal{I}_c} \coloneqq \left\{ v \in V \mid c(v) = \var{green}\right\}$
		\item $D^{\mathcal{I}_c} \coloneqq \emptyset$
	\end{itemize}
	Then $\mathcal{I}_c$ is a model of $(\mathcal{T}, \mathcal{A}_G)$,
	but $\mathcal{I}_c \not\vDash q$, which shows $(\mathcal{T}, \mathcal{A}_G) \not\vDash q$.

	"$ \impliedby$": By its contrapositive.
	Assume that $(\mathcal{T}, \mathcal{A}_G) \not\vDash q$.
	Then there is a model $\mathcal{I}$ of $(\mathcal{T}, \mathcal{A}_G)$ such that $\mathcal{I} \not\vDash q$.
	This means that $D^{\mathcal{I}} = \emptyset$.
	We define the mapping $c_{\mathcal{I}}: V \to \left\{ \var{red}, \var{blue}, \var{green} \right\}$ as follows:
	\[
		c_{\mathcal{I}}(v) \coloneqq
		\begin{cases}
			\var{red},& \text{if } v^\mathcal{I} \in R^\mathcal{I}; \text{ otherwise} \\
			\var{blue},& \text{if } v^\mathcal{I} \in B^\mathcal{I}; \text{ otherwise} \\
			\var{green},& \text{if } v^\mathcal{I} \in G^\mathcal{I} \\
		\end{cases}
	\]
	Claim: $c_{\mathcal{I}}$ is a 3-coloring of $G$.
	\begin{subproof}
		Let $\left\{ u,v \right\} \in E$, and assume that $c_{\mathcal{I}}(u) = c_{\mathcal{I}}(v) $.
		Then $(u^{\mathcal{I}}, v^{\mathcal{I}}) \in r^{\mathcal{I}}$ and $u^{\mathcal{I}}$ belongs to one of the concepts
		$\circ \sqcap \exists r.\circ$, where $\circ \in \left\{ R, B, G \right\}$.
		Consequently, $u^{\mathcal{I}} = D^{\mathcal{I}}$, which contradict the fact $D^{\mathcal{I}} = \emptyset$. \contra
	\end{subproof}
\end{proof}
