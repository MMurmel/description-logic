\newpage
\lecture{12}{01.06.2021}{}
\begin{lemma}[Completeness]
	If $ \mathcal{K}$ is consistent, then $\func{consistent}(\mathcal{K})$ returns "consistent".
\end{lemma}
\begin{proof}
	It only remains to show that the $\sqsubseteq$-rule preserves KB consistency.
	If $a :C \in \mathcal{A}$ and $\top \sqsubseteq D \in T$,
	then the $\sqsubseteq$-rule adds $a:D$.
	If $\mathcal{I}$ is a model of $\mathcal{A}$ and $\mathcal{T}$, then
	$a^\mathcal{I} \in \top^\mathcal{I}=\Delta^\mathcal{I}$ implies that $a^\mathcal{I} \in D^\mathcal{I}$ 
	since $\mathcal{I}$ satisfies $\top \sqsubseteq D$.
	Thus, $\mathcal{I}$ is also a model of the extended ABox.
\end{proof}

\begin{theorem}
	The tableau algorithm is a decision procedure for the consistency of $\mathcal{ALC}$ knowledge bases.
\end{theorem}

\newpage
\section{Tableau algorithm w.r.t.\ number restrictions}
One can also extend the tableau algorithm to deal with further constructors e.g.\ number restrictions as shown in the following.
Of course any concept description containing those must also be transformed into NNF:
\begin{itemize}
	\item $\neg \left( \geq n+1 r \right) \to ( \leq n r)$ 
	\item $\neg \left( \geq 0 r \right) \to \bot$
	\item $\neg \left( \leq n r \right) \to \left( \geq n + 1 r \right)$
\end{itemize}
And we will need to extend the algorithm by:
\begin{itemize}
	\item new rules: $ \geq$-rule and $ \leq$-rule
	\item new assertions: $x=y, x \neq y$ with the obvious semantics $x^\mathcal{I} = y^\mathcal{I}$ and $x^\mathcal{I} \neq y^\mathcal{I}$.
		These assertion are viewed as symmetric, because we do not want to distinguish $x = y$ and $y = x$.
	 \item a new clash
\end{itemize}

\begin{mdframed}[frametitle= The $ \geq$-rule]
	Condition: $\mathcal{A}$ contains $a:( \geq nr)$, but there are no distinct
	$b_1, \ldots, b_n$ with $\left\{ (a,b_1):r, \ldots, (a,b_n):r \right\} \subseteq \mathcal{A}$

	Action: $\mathcal{A} \leftarrow \mathcal{A} \cup \left\{ (a,d_1):r, \ldots, (a,d_n):r \right\} \cup \left\{ d_i \neq d_j \mid 1 \leq i < j \leq n \right\}$
	where $d_1, \ldots, d_n$ are new individual names.
\end{mdframed}
\begin{mdframed}[frametitle= The $ \leq$-rule, nobreak = true]
	Condition: $\mathcal{A}$ contains $a:( \leq nr)$, and there are distinct
	$b_0, \ldots, b_n$ with $\left\{ (a,b_0):r, \ldots, (a,b_n):r \right\} \subseteq \mathcal{A}$

	Action: $\mathcal{A} \leftarrow \mathcal{A}[b_j \mapsto b_i] \cup \left\{ b_i = b_j \right\}$
	for $i \neq j$ such that, if $b_j$ is a root individual, then so is $b_i$.
	$\mathcal{A}[b_j \mapsto b_i]$ means every occurrence of $b_j$ is replaced by $b_i$ in $\mathcal{A}$.
\end{mdframed}
\begin{note}
	While the $ \geq$-rule is deterministic, the $ \leq$-rule is again a non-deterministic rule.
\end{note}

\begin{mdframed}[frametitle= The new clash]
An ABox $\mathcal{A}$ contains a clash if
\[
\left\{ a:C, a: \neg C \right\} \subseteq \mathcal{A} \text{ or } \left\{ a \neq a \right\} \subseteq \mathcal{A}
\]
for some individual name $a$, and for some concept $C$.
\end{mdframed}
However, the algorithm introduced by this does not necessarily terminate.
E.g.\ the ABox shown in figure \ref{fig:yoyo-example} causes a problem and is known as the \textit{yo-yo example}.
	\begin{figure}[H]
		\centering
		\begin{subfigure}[t]{.475\textwidth}
			\centering
			\begin{tikzpicture}
				\node[default, label= right: {$P, ( \leq 1r),$},label= below right: {$ \exists r.P, \forall r.\exists r.P$}] (a) {$a$};
				\node[default, label= right: {$P$}, below of = a] (y) {$y$};
				\draw (a) edge[left] node{$r$} (y);
				\draw (a) edge[loop above, above] node{$r$} (a);
			\end{tikzpicture}
			\caption{Starting ABox}
		\end{subfigure}
		\hfill
		\begin{subfigure}[t]{.475\textwidth}
			\centering
			\begin{tikzpicture}
				\node[default, label= right: {$P, ( \leq 1r),$},label= below right: {$ \exists r.P, \forall r.\exists r.P$}] (a) {$a$};
				\node[default, label= right: {$P, \exists r.P$}, below of = a] (y) {$y$};
				\node[default, label= left: $P$, left of = y] (z) {$z$};
				\draw (a) edge[left] node{$r$} (y);
				\draw (a) edge[loop above, above] node{$r$} (a);
				\draw (y) edge[above] node{$r$} (z);
			\end{tikzpicture}
			\caption{The ABox after the application of the $\forall$-rule and $\exists$-rule.}
		\end{subfigure}
		\caption{From the ABox shown in (b), the $ \leq$-rule is applicable and yields the ABox on the left
		(with the only difference that the lower node is now named $z$).}
		This forms a cycle, showing non-termination of the algorithm.
		\label{fig:yoyo-example}
	\end{figure}

Furthermore, while our standard notion of blocking would be a solution in this example,
that does not hold in general, as shown in the lecture.
