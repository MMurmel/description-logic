\lecture{3}{20.04.2021}{}
\begin{example}
	Here is an example of an acyclic TBox:
	\begin{itemize}
		\item $ \var{Woman} \equiv \var{Person} \sqcap \var{Female}$
		\item $\var{Man} \equiv \var{Person} \sqcap \neg \var{Female}$
	\end{itemize}
	$\left\{ \var{Woman}, \var{Man} \right\} $ are the defined concepts and $\left\{ \var{Person}, \var{Female} \right\}$ are primitive concepts.
\end{example}

\begin{prop}
	For every acyclic TBox $\mathcal{T}$ we can effectively construct an equivalent acyclic TBox $ \widehat{\mathcal{T}}$
	such that the right-hand sides of concept definitions in $\widehat{\mathcal{T}}$ contain only primitive concepts.
\end{prop}
\begin{proof}
	By recursively expanding the (finitely many and acyclic) definitions.
\end{proof}

Given an acyclic TBox $\mathcal{T}$, a primitive interpretation $\mathcal{J}$ for $\mathcal{T}$ 
consists of a nonempty set $\Delta^{\mathcal{J}}$ together with an extension mapping $\cdot^{\mathcal{J}}$, that maps
\begin{itemize}
	\item primitive concepts $P$ to sets $P^{\mathcal{J}} \subseteq \Delta^{\mathcal{J}}$ 
	\item role names r to binary relations $r^\mathcal{J} \subseteq \Delta^{\mathcal{J}} \times \Delta^{\mathcal{J}}$
\end{itemize}
The interpretation $\mathcal{I}$ is an extension of the primitive interpretation $\mathcal{J}$ iff $\Delta^{\mathcal{J}} = \Delta^{\mathcal{I}}$ and
 \begin{itemize}
	\item $P^{\mathcal{J}} = P^{\mathcal{I}}$ for all primitive concepts $P$ 
	\item $r^{\mathcal{J}} = r^{\mathcal{I}}$ for all role names $r$
\end{itemize}

\begin{corollary}
	Let $\mathcal{T}$ be an acyclic TBox.
	Any primitive interpretation $\mathcal{J}$ for $\mathcal{T}$ has a unique extension to a model of $\mathcal{T}$.
\end{corollary}
\begin{proof}
	First construct the expanded TBox $\widehat{\mathcal{T}}$ for $\mathcal{T}$.
	Obviously, $\mathcal{T}$ and $\widehat{\mathcal{T}}$ have the same primitive and defined concepts.
	Let $\mathcal{J}$ be a primitive interpretation.
	We define its extension $\mathcal{I}$ as follows:
	For every defined concept $A$, let $A \equiv \hat{C}$ be its definition in $\widehat{\mathcal{T}}$.
	Since $\hat{C}$ contains only primitive concepts, $\hat{C}^{\mathcal{J}}$ is well-defined and we set $A^{\mathcal{I}} \coloneqq \hat{C}^{\mathcal{J}}$
	\begin{enumerate}
		\item $\mathcal{I}$ is a model of $\mathcal{T}$ : \newline
			By definition it is a model of $\widehat{\mathcal{T}}$ and we know $\mathcal{T}$ and $\widehat{\mathcal{T}}$ are equivalent.
		\item $\mathcal{I}$ is unique: \newline
			Assume that $\mathcal{I}'$ is another model of $\mathcal{T}$ that extends $\mathcal{J}$.
			Then $\mathcal{I}'$ is also a model of $\widehat{\mathcal{T}}$, and thus the following holds for all defined concepts $A$,
			where $A \equiv \hat{C}$ is the definition of $A$ in $\widehat{\mathcal{T}}$ :
			\[
				A^{\mathcal{I}'} = \hat{C}^{\mathcal{I}'} = \hat{C}^{\mathcal{J}} = \hat{C}^{\mathcal{I}} = A^{\mathcal{I}}
			. \qedhere \]
	\end{enumerate}
\end{proof}

\subsection*{Relationship with FOL}
$\mathcal{ALC}$-TBoxes can be translated into FOL:
\[
	\tau(\mathcal{T}) \coloneqq \bigwedge_{C \sqsubseteq D \in \mathcal{T}} \forall x.\left( \tau_x(C) \rightarrow \tau_x(D) \right)
.\]

\begin{lemma}
	Let $\mathcal{T}$ be a TBox, then $\mathcal{T}$ and $\tau(\mathcal{T})$ have the same models.
\end{lemma}
\begin{proof}
	trivial.
\end{proof}

\section{Assertional knowledge}
\subsection{$\mathcal{ALC}$ ABoxes}
While concepts make statements about sets of individuals, we now want to state knowledge about specific individuals.

\begin{definition}[Assertions and ABoxes]
	An assertion is of the form $a:C$ (concept assertion) or $(a,b):r$ (role assertion),
	where $C$ is a concept description, $r$ is a role and $a,b$ are individual names from a set $\mathscr{I}$ of such names (disjoint with $\mathscr{C} \uplus \mathscr{R}$).
	An ABox is a finite set of assertions.
	An Interpretation $\mathcal{I}$ is a model of an ABox $\mathcal{A}$ if it satisfies all its assertions:
	\begin{itemize}
		\item $a^{\mathcal{I}} \in C^{\mathcal{I}}$ for all $a:C \in \mathcal{A}$ 
		\item $(a^{\mathcal{I}}, b^{\mathcal{I}}) \in r^{\mathcal{I}}$ for all $(a,b):r \in \mathcal{A}$
	\end{itemize}
\end{definition}

\begin{example}
	An example of an ABox:
	\begin{itemize}
		\item $ \var{MAX}: \var{Student}$
		\item $(\var{MAX}, \var{DL}): \var{interested}$
	\end{itemize}
\end{example}

\subsection{Relationship with FOL}
$\mathcal{ALC}$-ABoxes can be translated into FOL:
\[
	\tau(\mathcal{A}) \coloneqq \bigwedge_{a:C \in \mathcal{A}} \tau_{x}(C)(a) \land \bigwedge_{(a,b):r \in \mathcal{A}} r(a,b)
.\]

\begin{lemma}
	Let $\mathcal{A}$ be an ABox and $\tau(\mathcal{A})$ its translation into FOL.
	Then $\mathcal{A}$ and $\tau (\mathcal{A})$ have the same models.
\end{lemma}
\begin{proof}
	trivial.
\end{proof}

\newpage
\section{Knowledge Bases}
We will now combine the TBox and the ABox to form the foundation for reasoning in an DL.
\begin{definition}
	A knowledge base $\mathcal{K} = (\mathcal{T}, \mathcal{A})$ consists of a TBox $\mathcal{T}$ and an ABox $\mathcal{A}$.
	The interpretation $\mathcal{I}$ is a model of the knowledge base $\mathcal{K}$, iff it is a model of its TBox and its ABox.
\end{definition}
The FOL-translation of a knowledge base is easy, given its components:
\[
\tau(\mathcal{K}) \coloneqq \tau(\mathcal{T}) \land \tau(\mathcal{A})
.\] 
\begin{lemma}
	Let $\mathcal{K}$ be a knowledge base and $\tau(\mathcal{K})$ its translation into FOL.
	Then $\mathcal{K}$ and $\tau(\mathcal{K})$ have the same models.
\end{lemma}

\subsection{Additional constructors}
Individual names can also be used as concept constructors.
\begin{example}[$\mathcal{O}$ in the naming scheme.]
	This allows \textbf{nominals} $\left\{ a \right\} $ for $a \in \mathscr{I}$ with the semantics
	\[
	\left\{ a \right\} ^{\mathcal{I}} \coloneqq \left\{ a^{\mathcal{I}} \right\} 
	.\] 
	In this case, the ABox can be expressed using GCIs:
	\begin{itemize}
		\item $a:C$ is expressed as $\left\{ a \right\} \sqsubseteq C$
		\item $(a,b):r$ is expressed as $\left\{ a \right\} \sqsubseteq \exists r.\left\{ b \right\} $
	\end{itemize}
\end{example}

\newpage
\section{Reasoning Problems and Services}
\begin{definition}[terminological reasoning]
	Let $\mathcal{T}$ be a TBox and $C,D$ be (possibly) complex concept descriptions.
	We then consider the following decision problems:
	\begin{enumerate}
		\item Satisfiability: $C$ is satisfiable w.r.t.\ $\mathcal{T}$ iff
			$C^{\mathcal{I}} \neq \emptyset$ for some model $\mathcal{I}$ of $\mathcal{T}$.
		\item Subsumption: $C$ is subsumed by $D$ w.r.t.\ $\mathcal{T}$ $\left( C \sqsubseteq_{\mathcal{T}} D \right)$ iff
			$C^{\mathcal{I}} \subseteq D^{\mathcal{I}}$ for all models $\mathcal{I}$ of $\mathcal{T}$.
		\item Equivalence: $C$ is equivalent to $D$ w.r.t.\ $\mathcal{T}$ $\left( C \equiv_{\mathcal{T}} D \right) $ iff
			$C^{\mathcal{I}} = D^{\mathcal{I}}$ for all models $\mathcal{I}$ of $\mathcal{T}$.
		\item TBox consistency: $\mathcal{T}$ is consistent iff it has a model.
	\end{enumerate}
\end{definition}
\begin{notation}
	In the book a slightly different notation is used:
	\begin{itemize}
		\item $\mathcal{T} \vDash C \sqsubseteq D$ instead of $C \sqsubseteq_{\mathcal{T}} D$
		\item $\mathcal{T} \vDash C \equiv D$ instead of $C \equiv_{\mathcal{T}}D$
	\end{itemize}
\end{notation}
\begin{note}
	If $\mathcal{T} = \emptyset$, then satisfiability/subsumption/equivalence w.r.t.\ $\mathcal{T}$ is simply called satisfiability/subsumption/equivalence
	and we just write $\sqsubseteq$ and $\equiv$.	
\end{note}

\begin{example}
	Some examples:
	\begin{itemize}
		\item $A \sqcap \neg A$ and $\forall r.A \sqcap \exists r.\neg A$ are not satisfiable (unsatisfiable) \newline
			(therefore they are also equivalent and equivalent to $\bot$).
		\item $A \sqcap B$ is subsumed by $A$ and by $B$.
		\item $\exists r.(A \sqcap B)$ is subsumed by $\exists r.A$ and by $\exists r.B$
		\item $\forall r.(A \sqcap B)$ is equivalent to $\forall r.A  \sqcap \forall r.B$
	\end{itemize}
\end{example}

\begin{lemma}
	Properties of subsumption:
	\begin{enumerate}
		\item The subsumption relation $\sqsubseteq_{\mathcal{T}}$ is a pre-order on concept descriptions, i.e.,
			\begin{itemize}
				\item $C \sqsubseteq_{\mathcal{T}}C$ (reflexive)
				\item $C \sqsubseteq_{\mathcal{T}} D \land D \sqsubseteq_{\mathcal{T}} E \implies C \sqsubseteq_{\mathcal{T}} E$ (transitive)
				\item It is however not antisymmetric in general and therefore not a partial order.
					This is because in general $C \sqsubseteq_{\mathcal{T}} D$ and $D \sqsubseteq_{\mathcal{T}} C$
					does imply $C \equiv_{\mathcal{T}} D$ but not necessarily $C = D$.
					They can be syntactically different descriptions for the same set.
			\end{itemize}
		\item The restriction constructors are monotonic w.r.t.\ subsumption, i.e.,
			\[
			C \sqsubseteq_{\mathcal{T}} D \implies \exists r.C \sqsubseteq_{\mathcal{T}} \exists r.D \land \forall r.C \sqsubseteq_{\mathcal{T}} \forall r.D
			.\] 
		\item Subsumption reasoning in monotonic, i.e., if $\mathcal{T} \subseteq  \mathcal{T}'$, then
			\[
				C \sqsubseteq_{\mathcal{T}} D \implies C \sqsubseteq_{\mathcal{T}'} D
			.\] 
	\end{enumerate}
\end{lemma}
