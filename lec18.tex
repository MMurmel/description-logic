\newpage
\section{Concept satisfiability in $\mathcal{ALCOI}$}
\lecture{18}{22.06.2021}{}
We will now have a look at some basic complexity results for $\mathcal{ALCOI}$.
Remember the semantics of the $\mathcal{O}$ and $\mathcal{I}$ constructor abbreviations:

$\mathcal{I}$ stands for inverse roles, i.e.\ if $r$ is a role, then for every interpretation $\mathcal{I}$:
\[
	\left( r^- \right)^\mathcal{I} \coloneqq \left\{ (e,d) \mid (d,e) \in r^\mathcal{I} \right\}
.\]
$\mathcal{O}$ is used for nominals, i.e.\ singleton sets of individuals 
with names $a$ from a set $\mathscr{I}$ of such names:
\[
	\left\{ a \right\}^\mathcal{I} \coloneqq \left\{ a^\mathcal{I} \right\}
.\]
Satisfiability w.r.t.\ general TBoxes in $\mathcal{ALCOI}$ remains in \textsc{ExpTime}.
This can be shown by adapting the type elimination algorithm (without proof here).

However, concept satisfiability in $\mathcal{ALCOI}$ is \textsc{ExpTime}-hard even without TBoxes.

\begin{example}
	Note, that the combination of inverse roles and nominal
	allows us to enforce infinite chains of role successors.

	Consider $C = \{ a \} \sqcap \exists u.\{ a \} \sqcap \forall u. \exists r.\exists u^-.\{ a \}$
	and a model $\mathcal{I}$ of $C$ :
	\begin{figure}[H]
		\centering
		\begin{tikzpicture}
			\node[default, label= above: $C$] (n1) {$a$};
			\draw (n1) edge[loop left, below] node{$u$} (n1);
			\node[default, label= above: $\exists u^-.\{a\}$, right of = n1] (n2) {};
			\node[default, label= above: $\exists u^-.\{a\}$, right of = n2] (n3) {};
			\node[right of = n3] (n4) {$\cdots$};
			\draw (n1) edge[above] node{$r$} (n2);
			\draw (n2) edge[above] node{$r$} (n3);
			\draw (n1) edge[above, bend right] node{$u$} (n2);
			\draw (n1) edge[above, bend right] node{$u$} (n3);
			\draw (n1) edge[above, bend right] node{$u$} (n4);
			\draw (n3) edge[above] node{$r$} (n4);
		\end{tikzpicture}
	\end{figure}
	Obviously, this model contains an infinite chain of $r$-successors,
	but that does not mean that every model $\mathcal{I}'$ of $C$ must contain infinitely many elements:
	\begin{figure}[H]
		\centering
		\begin{tikzpicture}
			\node[default, label= above: $C$] (n1) {$a$};
			\draw (n1) edge[loop left, left] node{$u$} (n1);
			\draw (n1) edge[loop right, right] node{$r$} (n1);
		\end{tikzpicture}
	\end{figure}
	However, in this cyclic case, an element of the concept $C$ would also have an infinite chain of $r$-successors.

	The role name $u$ is chosen on purpose,
	because it is a \textit{universal} role, i.e.\ it connects $a$ with every other element.
\end{example}
The combination of inverse roles and nominal allows us to enforce
that all elements of the domain are reachable via some role $u$ from some nominal $a$.

By reducing $\mathcal{ALC}$ concept satisfiability w.r.t.\ general TBoxes to $\mathcal{ALCOI}$ satisfiability,
we show that the latter is \textsc{ExpTime}-hard.
The idea of the reduction is that GCIs $C \sqsubseteq D$ can be propagated to all elements of the domain
via the universal role $u$, by adding a value restriction $\forall u.(\neg C \sqcup D)$ to $a$.

The reduction goes as follows:
Given an $\mathcal{ALC}$ concept $C_0$ and an $\mathcal{ALC}$ TBox $\mathcal{T}$, we define
\[
	D_0 = C_0 \sqcap \{a\} \sqcap \exists u.\{a\} \sqcap \forall u.\left( \bigsqcap_{C \sqsubseteq D \in \mathcal{T}} \neg C \sqcup D \right) \sqcap \forall u.\left( \bigsqcap_{i<k} \forall r_i.\exists u^-.\{a\} \right)
,\]
where $r_0, \ldots, r_{k-1}$ are all role names occurring in $C$ and $\mathcal{T}$ and their inverses,
and $u$ is a fresh role name.

It remains to show, that this reduction is correct, i.e.,
\[
C_0 \text{ is satisfiable w.r.t.\ } \mathcal{T} \iff D_0 \text{ is satisfiable}
.\]

\begin{proof}
	"$ \implies$":
	Let $\mathcal{I}$ be a model of $C_0$ w.r.t.\ $\mathcal{T}$ and let $d_0 \in C_0^\mathcal{I}$.
	We modify $\mathcal{I}$ as follows:
	\begin{itemize}
		\item $a^\mathcal{I} = d_0$ 
		\item $u^\mathcal{I} = \Delta^\mathcal{I} \times \Delta^\mathcal{I}$
	\end{itemize}
	It is easy to see that this modified interpretation is a model of $D_0$.

	"$ \impliedby$":
	Let $\mathcal{I}$ be a model of $D_0$, and $d_0 \in D_0^\mathcal{I}$.
	Let $\Delta^\mathcal{J}$ be the restriction of $\Delta^\mathcal{I}$ to those elements $d$ 
	such that $d$ is reachable from $d_0$
	by traveling an arbitrary number of times along the roles $r_0, \ldots, r_{k-1}$.
	Let $ \mathcal{J}$ be the restriction of $\mathcal{I}$ to this set.
	By induction on the structure of $\mathcal{ALCOI}$ concepts, we can easily prove:
	\begin{mdframed}
		For all $\mathcal{ALCOI}$ concepts $E$ that contain no nominals except $\left\{ a \right\}$ 
		and all $d \in \Delta^{\mathcal{J}}$, we have:
		\[
			d \in E^\mathcal{I} \iff d \in E^{\mathcal{J}}
		.\]
	\end{mdframed}
	\begin{subproof}
		By induction on the structure of $\mathcal{ALCOI}$ concepts:
		\begin{itemize}
			\item The cases of $E = E_1 \sqcap E_2, E = E_1 \sqcup E_2,$ and $E = \neg E_1$ are easy to show.
			\item $E = \exists r.E_1$:
				If $d \in E^\mathcal{I}$, then there is $e \in \Delta^\mathcal{I}$ with $(d,e) \in r^\mathcal{I}$ and $e \in E_1^\mathcal{I}$.
				Then $e \in \Delta^\mathcal{J}$ and thus $(d, e) \in r^\mathcal{J}$.
				Induction yields $e \in E_1^\mathcal{J}$.
				If $d \in E^\mathcal{J}$, then $(d,e)\in r^\mathcal{J}$ and $e \in E_1^\mathcal{J}$.
				Then clearly $(d,e) \in r^\mathcal{I}$ and $e \in E_1^\mathcal{I}$ by induction.
			\item $E = \forall r.E_1$: follows from $\exists$ and $\neg$.
				\qedhere
			\item $\exists = \left\{ b \right\}$ : but then $b=a$ and $d_0 = a^\mathcal{I} \in \Delta^\mathcal{J}$.
		\end{itemize}
	\end{subproof}
	By this claim, $d_0 \in D_0^\mathcal{J}$ and thus $d_0 \in C_0^\mathcal{J}$, i.e.\ $C_0 \neq \emptyset$.
	It remains to prove that $\mathcal{J}$ is a model of $\mathcal{T}$.
	Let $E \sqsubseteq F \in \mathcal{T}$ and $d \in E^\mathcal{J}$.
	By the claim $d \in E^\mathcal{I}$.
	Since $d$ is reachable from $d_0$ along the roles $r_0, \ldots, r_{k-1}$
	and by the definition of $D_0$, we have $d \in (\neg E \sqcup F)^\mathcal{I}$.
	This yields $d \in F^\mathcal{I}$, i.e.\ $\mathcal{T}$ is satisfied.
\end{proof}

% Baader counting
\stepcounter{definition}
\begin{theorem}
	In $\mathcal{ALCOI}$, concept satisfiability and subsumption (without TBoxes) is \textsc{ExpTime-hard}.
\end{theorem}

\section{Undecidable extensions of $\mathcal{ALC}$}
We will now focus on extensions of $\mathcal{ALC}$ that make reasoning undecidable.

The first one is a concept constructor, called \textit{role value maps}, that was already used in the first DL system \textsc{KL-ONE}.
It was thought that reasoning in this DL was in \textsc{P}, when it was later discovered to be undecidable.

\begin{definition*}[Role value maps]
	With role value maps a concept definition can be of the form
	\[
		\left( r_1 \circ \cdots \circ r_k \sqsubseteq s_1 \circ \cdots \circ s_l \right)
	,\]
	where $r_1, \ldots, r_k$ and $s_1, \ldots, s_l$ are role names.
	\begin{note}
		The $\sqsubseteq$-symbol here is not to be mistaken for the syntax of a GCI.
		This one is a concept constructor.
	\end{note}

	We define the semantics of role value maps as follows:
	\begin{align*}
		\begin{split}
			&\left( r_1 \circ \cdots \circ r_k \right)^\mathcal{I} (d_0) \coloneqq \\
			&\left\{ d_k \in \Delta^\mathcal{I} \mid \exists d_1,\ldots,d_{k-1}: (d_i, d_{i+1}) \in r_{i+1}^\mathcal{I} \text{ for } 0 \leq i < k \right\}
		\end{split}
		\\[3ex]
		\begin{split}
			&\left(r_1 \circ \cdots \circ r_k \sqsubseteq s_1 \circ \cdots \circ s_l \right)^\mathcal{I} \coloneqq \\
			&\left\{ d \in \Delta^\mathcal{I} \mid (r_1 \circ \cdots \circ r_k)^\mathcal{I}(d) \subseteq (s_1 \circ \cdots \circ s_l)^\mathcal{I}(d)\right\}
		\end{split}
	\end{align*}
\end{definition*}
\begin{note}
	Intuitively, by $\left( r_1 \circ \cdots \circ r_k \sqsubseteq s_1 \circ \cdots \circ s_l \right)$ we mean those  $d \in \Delta^\mathcal{I}$,
	where all the elements reachable from $d$ via the roles  $r_1, \ldots ,r_k$
	are a subset of the elements reachable via the roles $s_1, \ldots, s_l$.
\end{note}

\begin{example}
	Consider this TBox about universities:
	\begin{alignat*}{2}
		& \var{Course} &&\sqsubseteq \exists \var{held-at}.\var{University} \\
		& \var{Lecturer} &&\sqsubseteq \exists \var{teaches}.\var{Course} \sqcap \exists \var{employed-by}.\var{University}
	\end{alignat*}

	One could want to express, that the university at which the lecturer is employed is also the one, where he holds a course:
	\[
		\top \sqsubseteq \left( \var{teaches} \circ \var{held-at} \sqsubseteq \var{employed-by} \right)
	.\]
\end{example}
Unfortunately though, role value maps are not only very interesting, but also cause undecidability.
We first show undecidability in the presence of GCIs and later that the undecidability holds even without GCIs.

We reduce the tiling problem to satisfiability in $\mathcal{ALC}$ with role value maps.

%Baader
\stepcounter{definition}
\stepcounter{definition}
\stepcounter{definition}
\begin{definition}[tiling problem]
	A tiling problem is a triple $P= (T, H, V)$, where $T$ is a finite set of tile types
	and $H, V \subseteq T \times T$ represent the horizontal and vertical matching conditions.

	A mapping $\tau: \N \times \N \to T$ is a solution for $P$ if for all $i, j \geq 0$ the following holds:
	\begin{itemize}
		\item if $\tau(i,j) = t$ and $\tau(i+1,j) = t'$, then $(t,t') \in H$ ;
		\item if $\tau(i,j) = t$ and $\tau(i,j+1) = t'$, then $(t,t') \in V$.
	\end{itemize}

	The problem whether a given tiling problem $P$ has a solution is known to be undecidable.
\end{definition}
Given a tiling problem $P = (T, H, V)$, we construct a TBox $\mathcal{T}_P$ with role value maps
such that models of $\mathcal{T}_P$ represent solutions to P.
\begin{itemize}
	\item Concept names: for each tile $t \in T$ a concept name $A_t$ 
	\item Role names: $r_x$ and $r_y$ for the horizontal and vertical successor relations.
\end{itemize}
The TBox $\mathcal{T}_P$ consists of the following GCIs:
\begin{enumerate}[label=(\roman*)]
	\item Every position has a horizontal and a vertical successor:
		\[
			\top \sqsubseteq  \exists r_x .\top \sqcap \exists r_y. \top
		.\]
	\item Every position is labeled with exactly one tile type:
		\[
			\top \sqsubseteq \bigsqcup_{t \in T} A_t \sqcap \bigsqcap_{t,t' \in T, t \neq t'} \neg (A_t \sqcap A_{t'})
		.\]
	\item Adjacent tiles satisfy the matching conditions:
		\[
			\top \sqsubseteq \bigsqcup_{(t,t') \in H} (A_t \sqcap \forall r_x.A_{t'}) \sqcap \bigsqcup_{(t,t') \in V} (A_t \sqcap \forall r_y.A_{t'})
		.\]
	\item Every $r_x r_y$-successor is also a $r_y r_x$-successor and vice versa:
		\begin{align*}
			\top &\sqsubseteq (r_x \circ r_y \sqsubseteq r_y \circ r_x) \\
			\top &\sqsubseteq (r_y \circ r_x \sqsubseteq r_x \circ r_y)
		\end{align*}
\end{enumerate}
\newpage
\begin{lemma}[correctness of the reduction]
	$\top$ is satisfiable w.r.t.\ $\mathcal{T}_P \iff P$  has a solution.
\end{lemma}
