\lecture{23}{13.07.2021}{}
\begin{lemma}[completeness]
	Let $A,B \in \func{sig}(\mathcal{T})$ be such that $\left\{ A \right\}$ occurs in $\mathcal{T}^*$.
	Then $\mathcal{T} \vDash A \sqsubseteq B$ implies $\left\{ A \right\} \sqsubseteq \left\{ B \right\} \in \mathcal{T}^*$.
\end{lemma}
\begin{proof}
	We show the contrapositive:
	Assume that $\left\{ A \right\} \sqsubseteq \left\{ B \right\} \notin \mathcal{T}^*$.
	Since $\left\{ A \right\}$ occurs in $\mathcal{T}^*$, we have $\left\{ A \right\} \in \Delta^{\mathcal{I}_{\mathcal{T}^*}}$.
	Rule i.CR1 yields $\left\{ A \right\} \sqsubseteq \left\{ A \right\}$ 
	and thus $\left\{ A \right\} \in A^{\mathcal{I}_{\mathcal{T}^*}}$.
	However, $\left\{ A \right\} \sqsubseteq \left\{ B \right\} \notin \mathcal{T}^*$ yields
	$\left\{ A \right\} \notin B^{\mathcal{I}_{\mathcal{T}^*}}$.
	Since $\mathcal{I}_{\mathcal{T}^*}$ is a model of $\mathcal{T}^*$, and thus of $\mathcal{T}$,
	this yields $\mathcal{T} \not\vDash A \sqsubseteq B$.
\end{proof}

\begin{theorem}
	Subsumption in $\mathcal{ELI}$ w.r.t.\ general TBoxes is decidable in exponential time.
\end{theorem}
\begin{proof}
	Let $\mathcal{T}_0$ be a general $\mathcal{ELI}$ TBox and $C, D$ $\mathcal{ELI}$ concepts.
	To decide whether $\mathcal{T}_0 \vDash C \sqsubseteq D$, we first add the GCIs $A \sqsubseteq C, D \sqsubseteq B$ 
	to $\mathcal{T}_0$, where $A, B$ are fresh concept names.
	The resulting TBox $\mathcal{T}_1$ is then i.normalized using the normalization rules described in section 6.1.1
	together with the rule that transforms GCIs with an existential restriction on the left
	into equivalent GCIs with a value restriction on the right.
	The size of the i.normalized TBox $\mathcal{T}$ obtained this way is linear in the size of $\mathcal{T}_0$ 
	and we have $\mathcal{T}_0 \vDash C \sqsubseteq D \iff \mathcal{T} \vDash A \sqsubseteq B$.
	Let $\mathcal{T}^*$ be the TBox obtained from $\mathcal{T}$ by an exhaustive application of the classification rules starting with $\mathcal{T}' \subseteq \mathcal{T}$,
	where the tautological GCIs in $\mathcal{T}$ are removed, and the remaining ones are represented as $\mathcal{T}$-i.sequents.
	The i.saturated TBox $\mathcal{T}^*$ can be computed in exponential time, since there are only exponentially $\mathcal{T}$-i.sequents
	and every rule application adds one of them.
	Since $\mathcal{T}_0$ contains a GCI whose left-hand side is $A$, the initial set of $\mathcal{T}$-i.sequents contains $\left\{ A \right\}$.
	Thus, Lemma 6.25 yields
	\[
	\mathcal{T} \vDash A \sqsubseteq B \iff \left\{ A \right\} \sqsubseteq \left\{ B \right\} \in \mathcal{T}^*
	.\]
	Consequently, by checking whether $\left\{ A \right\} \sqsubseteq \left\{ B \right\} \in \mathcal{T}^*$,
	we can decide whether $\mathcal{T}_0 \vDash C \sqsubseteq D$ holds.
\end{proof}
Without proof, we state that this algorithm is actually optimal, since classification in $\mathcal{ELI}$ is \textsc{ExpTime}-hard.

\newpage
\subsection{Classification algorithm for $\mathcal{ELI}$ applied to $\mathcal{EL}$}
We can also show, that the algorithm for $\mathcal{ELI}$ runs in polynomial time if it receives a general $\mathcal{EL}$ TBox as input.
We define $\mathcal{EL}$-$\mathcal{T}$-i.sequents as $\mathcal{T}$-i.sequents satisfying the following restrictions:
\begin{itemize}
	\item the only sets occurring in them are the empty set and singleton sets;
	\item value restrictions in these $\mathcal{T}$-i.sequents are only w.r.t.\ inverses of role names;
	\item existential restrictions in these $\mathcal{T}$-i.sequents are only w.r.t.\ role names.
\end{itemize}
If we start with an $\mathcal{EL}$ TBox $\mathcal{T}_0$, then the corresponding i.normalized TBox $\mathcal{T}$ 
(written as a set of $\mathcal{T}$-i.sequents) contains only $\mathcal{EL}$-$\mathcal{T}$-i.sequents.

\begin{lemma}
	There are only polynomially many $\mathcal{EL}$-$\mathcal{T}$-i.sequents in the size of $\mathcal{T}$.
	
	In addition, applying a classification rule for $\mathcal{ELI}$ to a set $\mathcal{T}'$
	of $\mathcal{EL}$-$\mathcal{T}$-i.sequents yields a set of $\mathcal{EL}$-$\mathcal{T}$-i.sequents.
\end{lemma}
\begin{proof}
	The first part is easy to see.
	\newline
	The only rule that could generate a $\mathcal{T}$-i.sequent that is not a $\mathcal{EL}$-$\mathcal{T}$-i.sequent is i.CR4.
	However, this rule is not applicable since the existential restriction role names
	and the value restriction inverses of role names.
\end{proof}

\begin{prop}
	The subsumption algorithm for $\mathcal{ELI}$ yields a polynomial-time decision procedure for subsumption in $\mathcal{EL}$.
\end{prop}
\begin{proof}
	trivial.
\end{proof}

\chapter{Query answering}
While very useful in medicine, in many practical application computing a subsumption hierarchy is actually not of big interest.
In these circumstances query answering with certain "background knowledge" is a more relevant problem.

We first consider query answering in databases from a logical point of view,
and then extend this to ontology-mediated query answering in order to
\begin{itemize}
	\item allow for incomplete data;
	\item take background knowledge into account;
	\item deal with potentially infinite data sets.
\end{itemize}

In general, a database can be seen as a finite collection of relations over a finite domain.
We view the database as an interpretation $\mathcal{I}$ over a relational signature with a domain $\Delta^{\mathcal{I}}$ and interpretations for the relation symbols.
A SQL query describing answer tuples can be represented by a FOL formula  $\phi(x_1, \ldots, x_n)$ with the free variables representing the answer variables.
The result of this query are these variable assignments $\mathcal{V}$ for which $(\mathcal{I}, \mathcal{V}) \vDash \phi(x_1, \ldots, x_n)$.
It is known, that this logical view of databases is equivalent to the standard view (i.e.\ SQL is equivalent to FOL).

\section{FOL queries}
We can restrict our attention to queries with unary and binary relation symbols corresponding to concepts and roles in DLs.
Furthermore, all definitions are given for arbitrary interpretations, not just finite ones.

\begin{definition}[FO query]
	A FO query is a FO formula that uses only unary and binary predicate symbols, and no function symbols or constants.
	The use of equality is allowed.

	The free variables $\vec{x}$ of an FO query $q(\vec{x})$ are called answer variables.
	The arity of $q(\vec{x})$ is the number of answer variables.

	Let $q(\vec{x})$ be an FO query of arity $k$ and $\mathcal{I}$ an interpretation.
	We say that
	\[
		\vec{a} = a_1,\ldots,a_k \text{ is an answer to $q$ on  $\mathcal{I}$ if } \mathcal{I} \vDash q[\vec{a}]
	\]
	i.e., if $q(\vec{x})$ evaluates to $\var{true}$ in $\mathcal{I}$ under the valuation that interprets the answer variables $\vec{x}$ as the constants $\vec{a}$.

	We write $\func{ans}(q,\mathcal{I})$ for the set of all answers to $q$ in $\mathcal{I}$.
\end{definition}

\begin{definition}[conjunctive query]
	A conjunctive query (CQ) $q$ has the form
	\[
		\exists x_1 \cdots \exists x_k.(\alpha_1 \land \cdots \land \alpha_n)
	\]
	where $k \geq 0, n \geq 1, x_1, \ldots, x_k$ are variables, and
	each $\alpha_i$ is a concept atom $A(x)$ or a role atom $r(x,y)$ 
	with $A \in \mathscr{C}, r \in \mathscr{R}$, and $x,y$ variables.

	We call $x_1, \ldots, x_k$ the quantified variables and all other variables in $q$ the answer variables.
	The arity of $q$ is the number of answer variables.
	
	To express that the answer variables in a CQ $q$ are $\vec{x}$,
	we often write $q(\vec{x})$ instead of just $q$.
\end{definition}

\begin{definition}[$\vec{a}$-math]
	Let $q$ be a conjunctive query and $\mathcal{I}$ an interpretation.
	We use $\func{var}(q)$ to denote the set of variables in $q$.

	A match of $q$ in $\mathcal{I}$ is a mapping $\pi: \func{var}(q) \to \Delta^{\mathcal{I}}$ such that
	\begin{itemize}
		\item $\pi(x) \in A^{\mathcal{I}}$ for all concept atoms $A(x)$ in $q$, and
		\item $\left( \pi(x),\pi(y) \right) \in r^{\mathcal{I}}$ for all role atoms $r(x,y)$ in $q$.
	\end{itemize}
	Let $\vec{x} = x_1, \ldots, x_k$  be the answer variables in $q$ and
	$\vec{a} = a_1, \ldots, a_k$ individual names from $\mathscr{I}$.

	We call the match $\pi$ of $q$ in $\mathcal{I}$ an $\vec{a}$-math if $\pi(x_i) = a_i^{\mathcal{I}}$ for $1 \leq i \leq k$.
\end{definition}

\begin{lemma}
	\[
		\func{ans}(q, \mathcal{I}) = \left\{ \vec{a} \mid \text{there is an $\vec{a}$-match of $q$ in $\mathcal{I}$} \right\}
	.\]
\end{lemma}
