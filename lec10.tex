\lecture{10}{18.05.2021}{}
\begin{lemma}
	If $\mathcal{A}$ is consistent, then $\func{consistent}(\mathcal{A})$ returns "consistent".
\end{lemma}
\begin{proof}
	Let $\mathcal{A}$ be consistent and consider a model $\mathcal{I} = (\Delta^\mathcal{I}, \cdot^\mathcal{I})$ of $\mathcal{A}$.
	Since $\mathcal{A}$ is consistent, it cannot contain a clash.
	Thus, if $\mathcal{A}$ is complete, then $\func{expand}(\mathcal{A})$ simply returns $\mathcal{A}$ and $\func{consistent}(\mathcal{A})$ returns "consistent".
	If $\mathcal{A}$ is not complete, then $\func{expand}$ calls itself recursively until $\mathcal{A}$ is complete;
	each call selects a rule and applies it.
	It is thus sufficient to show that rule application preserves consistency.
	\begin{itemize}
		\item The $\sqcup$-rule: If $a : C \sqcup D \in \mathcal{A}$, then $a^\mathcal{I} \in (C \sqcup D)^\mathcal{I} = C^\mathcal{I} \cup D^\mathcal{I}$,
			i.e.\ $a^\mathcal{I} \in C^\mathcal{I}$ or $a^\mathcal{I} \in D^\mathcal{I}$.

			Therefore, at least one of the ABoxes $\mathcal{A}' \in \func{exp}(\mathcal{A}, \sqcup \text{-rule}, a:C \sqcup D)$ is consistent.
			Thus, one of the calls of $\func{expand}$ is applied to a consistent ABox.
		\item The $\sqcap$-rule: If $a : C \sqcap D \in \mathcal{A}$, then $a^\mathcal{I} \in (C \sqcap D)^\mathcal{I} = C^\mathcal{I} \cap D^\mathcal{I}$ 
			and thus $a^\mathcal{I} \in C^\mathcal{I}$ and $a^\mathcal{I} \in D^\mathcal{I}$.
			Thus, $\mathcal{I}$ is still a model of $\mathcal{A} \cup \left\{ a:C, a:D \right\}$, so the ABox is still consistent after application of the $\sqcap$-rule.
		\item The $\exists$-rule: If $a : \exists r.C \in \mathcal{A}$, then $a^\mathcal{I} \in (\exists r.C)^\mathcal{I}$.
			Thus, there is $x \in \Delta^\mathcal{I}$ such that $(a^\mathcal{I}, x) \in r^\mathcal{I}$ and $x \in C^\mathcal{I}$.
			Application of the $\exists$-rule to $a : \exists r.C$ yields a new individual $d$ such that $(a,d) \in r^\mathcal{I}$ and $d : C$ are added to the ABox.
			If we modify  $\mathcal{I}$ such that $d^\mathcal{I} \coloneqq x$, then $\mathcal{I}$ is still a model of the extended ABox.
			Thus, this ABox is still consistent.
		\item The $\forall$-rule: If $a : \forall r.C \in \mathcal{A}$ and $(a,b):r \in \mathcal{A}$, then $a^\mathcal{I} in (\forall r.C)^\mathcal{I}$ and $(a^\mathcal{I}, b^\mathcal{I) \in r^\mathcal{I}}$,
			which implies $b^\mathcal{I} \in C^\mathcal{I}$.
			Thus, the ABox $\mathcal{A} \cup \left\{ b:C \right\}$ obtained by applying the $\forall$-rule still has $\mathcal{I}$ as a model,
			and thus is consistent.
			\qedhere
	\end{itemize}
\end{proof}

\begin{theorem}
	The tableau algorithm is a decision procedure for the consistency of  $\mathcal{ALC}$-ABoxes.
\end{theorem}
We will see in Chapter 5 that the complexity of the $\mathcal{ALC}$-ABox consistency problem is \textsc{PSPACE}-complete.
However, the algorithm presented here needs exponential time and space.
It could be modified such that it uses only polynomial space, but we will not do this here in detail.
For a general idea see the lecture slides.

\subsubsection{Tableau algorithm w.r.t.\ acyclic TBoxes}
We will now look briefly at a tableau algorithm w.r.t.\ acyclic TBoxes.
Like we have already seen, this is not necessary to obtain a valid decision procedure,
because consistency of ABoxes w.r.t.\ acyclic TBoxes can be reduced to consistency of ABoxes without TBoxes by expansion.
However, this expansion w.r.t.\ to an acyclic TBox may result in an exponential blow-up, rendering it useless for efficient computation.
To solve this, we will only do this expanding/unfolding as needed, a technique called \textit{lazy unfolding}.
\begin{mdframed}[frametitle= The $ \equiv_1$-rule]
	Condition: $a : A \in \mathcal{A}, A  \equiv C \in \mathcal{T}$ and $a:C \notin \mathcal{A}$ \\
	Action: $\mathcal{A} \leftarrow \mathcal{A} \cup \left\{ a:C \right\}$
\end{mdframed}
\begin{mdframed}[frametitle= The $ \equiv_2$-rule, nobreak = true]
	Condition: $a : \neg A \in \mathcal{A}, A \equiv C \in \mathcal{T}$ and $a : \dot{\neg} C \notin \mathcal{A}$ \\
	Action: $\mathcal{A} \leftarrow \mathcal{A} \cup \left\{ a: \dot{\neg}C \right\}$
\end{mdframed}
In the $\equiv_2$-rule, $\dot{\neg}C$ means the NNF of $\neg C$.
Termination, soundness and completeness for the hereby extended algorithm can be shown similarly to the case without TBox (Exercise).
Furthermore, this algorithm can be written to run in polynomial space (without proof).

\subsubsection{Tableau algorithm w.r.t.\ general TBoxes}
To realize a tableau algorithm w.r.t.\ general TBoxes we will first need some preprocessing.
We will also normalize the TBox:
\begin{itemize}
	\item transform al GCIs in $\mathcal{T}$ to the form $\top \sqsubseteq E$.
		This transformation is justified, by the following statement:
		\[
		\mathcal{I} \text{ satisfies } C \sqsubseteq D \iff \mathcal{I} \text{ satisfies } \top \sqsubseteq D \sqcup \neg C
		.\]
	\item transform the right-hand sides $E$ of GCIs $\top \sqsubseteq E$ into NNF
\end{itemize}
In the following, we assume that the input TBox $\mathcal{T}$ is normalized in this way.

We add a new expansion rule:
\begin{mdframed}[frametitle= The $\sqsubseteq$-rule, nobreak = true]
	Condition: $a : C \in \mathcal{A}, \top \sqsubseteq D \in \mathcal{T}, a : D \notin \mathcal{A}$\\
	Action: $\mathcal{A} \leftarrow \mathcal{A} \cup \left\{ a: D \right\}$
\end{mdframed}
\begin{note}
	This rule ensures, that all individuals $a$ are explicitly stated as belonging to the right hand sides of $\top \sqsubseteq D$.
	Since the input ABox is normalized, all individuals do occur in concept assertions (and not just in role assertions).
\end{note}

Soundness and completeness of the tableau algorithm extended with this rule is relatively easy to show,
however, this rule may even destroy termination,
e.g.\ with input $(\left\{ A \sqsubseteq \exists r.A \right\}, \left\{ a :A \right\})$, it could just create an infinite chain of individuals $a_i$ belonging to  $A$.
