\lecture{7}{04.05.2021}{}

\begin{lemma} \label{lem:card(sub)<=size}
	For arbitrary concept descriptions $C$ and arbitrary TBoxes $ \mathcal{T}$ the following hold:
	\begin{itemize}
		\item $\left| \text{sub}(C) \right| \leq \text{size}(C)$ 
		\item $\left| \text{sub}(\mathcal{T}) \right| \leq \text{size}(\mathcal{T})$
	\end{itemize}
\end{lemma}

Furthermore, we need the notion of a so called \textit{$S$-type}.
This will enable us to define an interesting equivalence relation, that is useful in proving the finite model property of $\mathcal{ALC}$.
\begin{definition}[$S$-type]
	Let $S$ be a finite set of concept descriptions and $\mathcal{I}$ an interpretation.
	We define the $S$-type of $d \in \Delta^{\mathcal{I}}$ as
	\[
		t_S(d) \coloneqq \left\{ C \in S \mid d \in C^{\mathcal{I}} \right\}
	.\]
\end{definition}

\begin{lemma}[number of $S$-types]\label{lem:number of s-types}
	Let $S$ be a finite set of concept descriptions and $\mathcal{I} = (\Delta^{\mathcal{I}}, \cdot^{\mathcal{I}})$ an interpretation.
	Then we have:
	 \[
		 \left| \left\{ t_S(d) \mid d \in \Delta^{\mathcal{I}} \right\} \right| \leq 2^{\left| S \right|}
	.\]
\end{lemma}
\begin{proof}
	$S$-types are subsets of $S$, i.e.\ elements of the power set $\mathscr{P}(S)$,
	of which there are only $\lvert \mathscr{P}(S) \rvert = 2^{\lvert S \rvert}$.
\end{proof}

\begin{definition}[$S$-filtration]
	Let $S$ be a finite set of concept descriptions and $\mathcal{I}$ an interpretation.
	We define an equivalence relation $\simeq$ on $\Delta^{\mathcal{I}}$ as follows:
	\[
		d \simeq e \iff t_S(d) = t_S(e)
	.\]
	The $\simeq$-equivalence class of  $d \in \Delta^{\mathcal{I}}$ is denoted by $\left[ d \right]$.
	The $S$-filtration of $\mathcal{I}$ is the following interpretation $\mathcal{J}$ :
	\begin{align*}
		\Delta^{\mathcal{J}} &\coloneqq \left\{ [d] \mid d \in \Delta^{\mathcal{I}} \right\} \\
		A^\mathcal{J} &\coloneqq \left\{ [d] \mid \exists d' \in [d]. d' \in A^{\mathcal{I}} \right\} \text{ for all } A \in \mathscr{C} \\
		r^\mathcal{J} &\coloneqq \left\{ \left( [d],[e] \right) \mid \exists d' \in [d], e' \in [e]. (d',e') \in r^\mathcal{I}\right\} \text{ for all } r \in \mathscr{R}
	\end{align*}
\end{definition}
\begin{note}
	By Lemma \ref{lem:number of s-types}, $\lvert \Delta^{\mathcal{J}} \rvert \leq 2^{\lvert S \rvert}$.
	This gives us an upper bound on the size of a model.
	Obviously, we want this $S$-filtration to behave similar to the interpretation $\mathcal{I}$.
	For this to work we require the additional notion of closure (under subconcepts).
\end{note}

We say that the finite set $S$ of concept descriptions is \textbf{closed} iff
\[
	\bigcup_{} \left\{ \text{sub}(C) \mid C \in S \right\} \subseteq S
.\]
This will enable us to use induction on $S$, because whenever a concept is in $S$, then all of its subconcepts are also in $S$.

\begin{lemma} \label{lem:filtration invariance}
	Let $S$ be a finite, closed set of $\mathcal{ALC}$-concept descriptions,
	$\mathcal{I}$ an interpretation, and $\mathcal{J}$ the $S$-filtration of $\mathcal{I}$.
	Then we have
	\[
		d \in C^{\mathcal{I}} \iff [d] \in C^{\mathcal{J}}
	\]
	for all $d \in \Delta^{\mathcal{I}}$ and $C \in S$.
\end{lemma}
\begin{proof}
	By induction on the structure of $C$,
	where we restrict the attention to concept names, conjunction, and existential restriction.
	\begin{enumerate}
		\item Assume that $C=A$ is a concept name.
			\begin{itemize}
				\item "$\implies$": If $d \in A^{\mathcal{I}}$, then $[d] \in A^{\mathcal{J}}$ by definition of $\mathcal{J}$.
				\item "$\impliedby$": If $\left[ d \right] \in A^{\mathcal{J}}$, then there is $d' \in [d]$ with $d' \in A^{\mathcal{I}}$.
					Since $d \simeq d'$ and $A \in S$, $d' \in A^{\mathcal{I}}$ implies $d \in A^{\mathcal{I}}$.
			\end{itemize}
		\item Assume that $C = D \sqcap E$. Then:
			\begin{align*}
				d \in (D \sqcap E)^{\mathcal{I}} &\iff d \in D^\mathcal{I} \land d \in E^{\mathcal{I}} \\
												 &\stackrel{\text{I.H.}}{\iff} [d] \in D^{\mathcal{J}} \land [d] \in E^{\mathcal{J}} \\
												 &\iff [d] \in (D \sqcap E)^{\mathcal{J}}
			\end{align*}
			$C = \neg D$ and $C = D \sqcup E$ can be treated similarly.
		\item Assume that $C = \exists r.D$.
			Since $S$ is closed, $D \in S$ and we can use the induction hypotheses.
			\begin{itemize}
				\item "$\implies$": If $d \in (\exists r.D)^{\mathcal{I}}$, then there is $e \in \Delta^{\mathcal{I}}$ s.t. $(d,e) \in r^{\mathcal{I}}$ and $e \in C^{\mathcal{I}}$.
					Then $([d],[e]) \in r^{\mathcal{J}}$ and $[e] \in C^\mathcal{J}$.
					This shows that $[d] \in (\exists r.D)^\mathcal{J}$.
				\item "$\impliedby$": If $[d] \in (\exists r.D)^{\mathcal{J}}$, then there is $[e] \in \Delta^{\mathcal{J}}$ s.t. $([d],[e]) \in r^{\mathcal{J}}$ and $[e] \in D^{\mathcal{J}}$.
					Induction yields $e \in D^\mathcal{I}$.
					In addition, we have $d' \in [d]$ and $e' \in [e]$ such that $(d', e') \in r^{\mathcal{I}}$.
					Since $e \simeq e'$ and  $D \in S$, we know $e \in D^{\mathcal{I}} \implies e' \in D^{\mathcal{I}}$.
					This yields $d' \in \left( \exists r.D \right)^\mathcal{I}$.
					But then $d \simeq d'$ and  $\exists r.D \in S$ yields $d \in \left( \exists r.D \right)^\mathcal{I}$.
					\qedhere
			\end{itemize}
	\end{enumerate}
\end{proof}
Unfortunately, this proof is not enough to immediately show the finite model property of $\mathcal{ALC}$.
Although one might think that what was shown here is very similar to the bisimulation invariance shown earlier,
it is however not sufficient in this case.

The following proposition shows that $\mathcal{ALC}$ satisfies a property that is even stronger than the finite model property.
\begin{theorem}[bounded model property]
	Let $\mathcal{T}$ be an $\mathcal{ALC}$-TBox, $C$ an $\mathcal{ALC}$-concept description, and $n = \text{size}(\mathcal{T}) + \text{size}(C)$.
	If $C$ has a model w.r.t.\ $\mathcal{T}$, then it has a model $\widehat{\mathcal{I}}$ s.t. $\lvert \Delta^{\widehat{\mathcal{I}}} \rvert \leq 2^n$.
\end{theorem}
\begin{proof}[Proof sketch]
	Let $\mathcal{I}$ be a model of $\mathcal{T}$ with $C^{\mathcal{I}} \neq \emptyset$ and $\widehat{\mathcal{I}}$ be the $S$-filtration of $\mathcal{I}$,
	where $S \coloneqq \text{sub}(C) \cup \text{sub}(\mathcal{T})$.
	We then show:
	\begin{enumerate}
		\item $\lvert \Delta^{\widehat{\mathcal{I}}} \rvert \leq 2^{n}$, by lemma \ref{lem:card(sub)<=size} and \ref{lem:number of s-types},
		\item $C^{\widehat{\mathcal{I}}} \neq \emptyset$, by lemma \ref{lem:filtration invariance},
		\item $\widehat{\mathcal{I}}$ is a model of $\mathcal{T}$, also by lemma \ref{lem:filtration invariance}.
	\end{enumerate}
\end{proof}

\begin{corollary}[Finite model property]
	Let $\mathcal{T}$ be an $\mathcal{ALC}$-TBox and $C$ an $\mathcal{ALC}$-concept description.
	If $C$ has a model w.r.t.\ $\mathcal{T}$, that it has a finite model.
\end{corollary}

\begin{corollary}[Decidability]
	In $\mathcal{ALC}$, satisfiability of a concept description w.r.t.\ a TBox is decidable.
\end{corollary}
\begin{proof}
	Let $n \coloneqq \text{size}(T) + \text{size}(C)$.
	If $C$ is satisfiable w.r.t.\ $\mathcal{T}$, then there is a model of cardinality at most $2^{n}$.
	Up to isomorphism (i.e.\, up to renaming of the domain elements), there are only finitely many interpretations satisfying this bound.
	Thus, we can enumerate all these interpretations and check whether the are models of $C$ w.r.t.\ $\mathcal{T}$.
\end{proof}

\begin{theorem}
	$\mathcal{ALCIN}$ does not have the finite model property.
\end{theorem}
\begin{proof}[Proof sketch]
	Let $C = \neg A \sqcap \exists r.A$ and $\mathcal{T} = \left\{ A \sqsubseteq \exists r.A, \top \sqsubseteq ( \leq 1r^-) \right\}$.
	Then the following interpretation is a model of $C$ w.r.t.\ $\mathcal{T}$:
	\begin{figure}[H]
		\centering
		\begin{tikzpicture}
			\node[default, label= below: $\neg A$] (d0) {$d_0$};
			\node[default, label= below: $A$, right of = d0] (d1) {$d_1$};
			\node[default, label= below: $A$, right of = d1] (d2) {$d_2$};
			\node[right of = d2] (dd) {$\cdots$};
			\node[default, label= below: $A$, right of = dd] (dn) {$d_n$};
			\node[right of = dn] (dd2) {$\cdots$};
			\draw (d0) edge[above] node{$r$} (d1);
			\draw (d1) edge[above] node{$r$} (d2);
			\draw (d2) edge[above] node{$r$} (dd);
			\draw (dd) edge[above] node{$r$} (dn);
			\draw (dn) edge[above] node{$r$} (dd2);
		\end{tikzpicture}
		\caption{}
		\label{}
	\end{figure}
	It is quite intuitive, that there is no smaller model of $C$ w.r.t.\ $\mathcal{T}$,
	since no element $d \in \Delta^{\mathcal{I}}$ can be "reused" for a cycle (e.g.\ from $d_n$).
	$d_0$ cannot be used, because it is not in $A$, and any other $d_i$ with $i \geq 1$ is prevented, because $\top \sqsubseteq ( \leq 1r^-) \in  \mathcal{T}$.
\end{proof}

\newpage
\section{Tree model property}
We will now show an other important property of $\mathcal{ALC}$, the \textit{Tree model property},
i.e.\ whenever a model is satisfiable it has a model, whose graph is a tree.
\begin{example}[Intuition]
	Consider the following interpretation $\mathcal{I}$ :
	\begin{figure}[H]
		\centering
		\begin{tikzpicture}[node distance = 1.5cm]
			\node[default, label= above: $\left\{ A \right\}$] (d) {$d$};
			\node[default, label= left: $\left\{ B \right\}$, below left of = d] (e) {$e$};
			\node[default, label= right: $\emptyset$, below right of = d] (f) {$f$};
			\draw (e) edge[above, bend left] node{$r$} (d);
			\draw (d) edge[above, bend left] node{$r$} (e);
			\draw (d) edge[above] node{$s$} (f);
			\draw (e) edge[below] node{$s$} (f);
		\end{tikzpicture}
	\end{figure}
	Obviously, $\mathcal{I}$ is a model of the knowledge base $\mathcal{K}=(\mathcal{T},\mathcal{A})$ with
	\begin{align*}
		\mathcal{T} = \big\{ A &\sqsubseteq \exists r.B, \\
					B &\sqsubseteq \exists r.A, \\
					A \sqcup B &\sqsubseteq \exists s.\top \big\} \\
	\end{align*}
	and $\mathcal{A} = \left\{ d:A \right\}$.\\
	We can then "unravel" $\mathcal{I}$ and get the following interpretation $\widehat{\mathcal{I}}$, 
	that is a tree model for $\mathcal{K}$ :
	\begin{figure}[H]
		\centering
		\begin{tikzpicture}[node distance = 1.5cm]
			\node[default, label= above left: $\left\{ A \right\}$] (d0) {$d_0$};
			\node[default, label= above left: $\left\{ B \right\}$, below left of = d0] (e0) {$e_0$};
			\node[default, label= right: $\emptyset$, below right of = d0] (f0) {$f_0$};
			\node[default, label= above left: $\left\{ A \right\}$, below left of =e0] (d1) {$d_1$};
			\node[default, label= right: $\emptyset$, below right of = e0] (f1) {$f_1$};
			\node[default, label= above left: $\left\{ B \right\}$, below left of = d1] (e1) {$e_1$};
			\node[default, label= right: $\emptyset$, below right of = d1] (f2) {$f_2$};
			\node[below left of =e1] (dd) {$\cdots$};
			\node[below right of =e1] (dd2) {$\cdots$};
			\draw (d0) edge[above] node{$s$} (f0);
			\draw (d0) edge[above] node{$r$} (e0);
			\draw (e0) edge[above] node{$s$} (f1);
			\draw (e0) edge[above] node{$r$} (d1);
			\draw (d1) edge[above] node{$r$} (e1);
			\draw (d1) edge[above] node{$s$} (f2);
			\draw (e1) edge[above] node{$r$} (dd);
			\draw (e1) edge[above] node{$s$} (dd2);
		\end{tikzpicture}
	\end{figure}
\end{example}

\begin{definition}[Tree model]\label{def:tree model}
	Let $\mathcal{T}$ be a TBox and $C$ be a concept description.
	The interpretation $\mathcal{I}$ is a tree model of $C$ w.r.t.\ $\mathcal{T}$ iff
	\begin{itemize}
		\item $\mathcal{I}$ is a model of $\mathcal{T}$ and
		\item the graph 
			\[
				\mathcal{G}_{\mathcal{I}} = \left( \Delta^{\mathcal{I}}, \bigcup_{r \in \mathscr{R}} r^{\mathcal{I}} \right)
			\]
			is a tree whose root belongs to $C^{\mathcal{I}}$.
	\end{itemize}
\end{definition}

Furthermore, the "unraveling" step needs to be formalized.
For this we will use the following terminology:

Let $\mathcal{I}$ be an interpretation and $d \in \Delta^{\mathcal{I}}$.
A $d$-path in $\mathcal{I}$ is a finite sequence $p = d_0,d_1,\ldots,d_{n-1}$ of $n \geq 1$ elements of $\Delta^{\mathcal{I}}$ such that
\begin{itemize}
	\item $d_0 = d$,
	\item $\forall i, 1 \leq i < n .\exists r_i \in \mathscr{R}: (d_{i-1},d_i) \in r_{i}^{\mathcal{I}}$.
\end{itemize}
We then call $n$ the length of this path and denote by $\text{end}(p) = d_{n-1}$ its end node.

\begin{definition}[Unraveling]\label{def:unraveling}
	The unraveling of $\mathcal{I}$ at $d$ is the following interpretation $\mathcal{J}$ :
	\begin{align*}
		\Delta^{\mathcal{J}} &= \left\{ p \mid p \text{ is a $d$-path in $\mathcal{I}$} \right\}, \\
		A^{\mathcal{J}} &= \left\{ p \in \Delta^{\mathcal{J}} \mid \text{end}(p) \in A^{\mathcal{I}} \right\} \text{for all } A \in \mathscr{C} \\
		r^{\mathcal{J}} &= \left\{ (p,p') \in \Delta^{\mathcal{J}} \times \Delta^{\mathcal{J}} \mid p'=(p, \text{end}(p')) \land (\text{end}(p),\text{end}(p')) \in r^{\mathcal{I}} \right\} \\
						&\text{for all } r \in \mathscr{R}
	\end{align*}
\end{definition}
\begin{note}
	While $\Delta^{\mathcal{J}}$ and $A^{\mathcal{J}}$ can be understood quite easily, $r^{\mathcal{J}}$ might need further explanation:
	When defining the role relations for the tree, we take the ends of a path $p$
	and extend it by a single element $\text{end}(p')$. If the tuple $(\text{end}(p), \text{end}(p'))$ then is in the original interpretation,
	we add it to the tree.
\end{note}

\begin{lemma}\label{lem:tree model bisimulation}
	Let $\mathcal{I}$ be an interpretation and $\mathcal{J}$ the unraveling of $\mathcal{I}$ at any $d \in \Delta^{\mathcal{I}}$.
	Then the relation
	\[
		\rho = \left\{ (p,\text{end}(p)) \mid p \in \Delta^{\mathcal{J}} \right\}
	\]
	is a bisimulation between $\mathcal{J}$ and $\mathcal{I}$.
\end{lemma}
