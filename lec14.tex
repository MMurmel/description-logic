\lecture{14}{08.06.2021}{}
\subsection{Soundness}
Soundness can actually be show similarly to the proof of Lemma 4.11.
However, the construction of $\mathcal{A}''$ needs to be modified
to obtain a complete and clash-free ABox.

Instead of just reusing blocking predecessors,
one would need to copy blocking predecessors for every successors it blocks.

\subsection{Completeness}
It only remains to show that the $ \leq$- and $ \geq$-rule preserve KB consistency.
\begin{proof}
	Exercise!
\end{proof}

\chapter{Complexity}
As common with complexity considerations,
we will not analyze the complexity of a particular algorithm,
but the complexity of the reasoning problem itself.

We know these complexity classes:
\[
	\textsc{P} \subseteq \textsc{NP} \subseteq \textsc{PSpace} \subseteq \textsc{ExpTime} \subseteq \textsc{NExpTime}
.\]
It is not known if these inclusions are strict or not.
The only one certainly known is, that $\textsc{P} \subset \textsc{ExpTime}$.
We will also use the notions of a problem being
\begin{itemize}
	\item \textit{hard} for a class \textsc{C} : every problem in \textsc{C} can be reduced to it in polynomial time
	\item \textit{complete} for a class \textsc{C}: hard for \textsc{C} and contained in \textsc{C} 
\end{itemize}

In the following we will concentrate on the basic reasoning problems satisfiability subsumption.
All results established in this chapter also apply to KB consistency.

\newpage
\section{Concept satisfiability in $\mathcal{ALC}$}

We will show, that concept satisfiability in $\mathcal{ALC}$
\begin{itemize}
	\item with acyclic TBoxes is in \textsc{PSpace}
	\item without TBoxes is \textsc{PSpace}-hard
\end{itemize}
and thus, \textsc{PSpace}-complete in both cases.

Furthermore, we will only concentrate on the complexity of satisfiability,
since it immediately yields the complexity of subsumption:
\begin{itemize}
	\item polynomial-time reductions between satisfiability and \textit{non}-subsumption
	\item for deterministic complexity classes \textsc{C} (such as \textsc{PSpace})
		we know that $ \text{co-\textsc{C}} = \textsc{C}$
\end{itemize}

\subsection{complexity upper bound (case of acyclic TBox)}
W.l.o.g.\ we assume that the concept tested for satisfiability is a concept name:

$C$ is satisfiable w.r.t.\ a TBox $\mathcal{T} \iff A$ is satisfiable w.r.t.\ $\mathcal{T} \cup \left\{ A \equiv C \right\}$,
where $A$ is a fresh concept name.

An acyclic TBox $\mathcal{T}$ is in NNF if negation is applied only to primitive concept names in $\mathcal{T}$,
but neither to defined concept names nor to compound concepts.

\begin{prop}[NNF]\label{prop:5.1}
	There is a polynomial time transformation of each acyclic TBox $\mathcal{T}$
	into an acyclic TBox $\mathcal{T}'$ in NNF such that  for all concept names $A$ occurring in $\mathcal{T}$,
	$A$ is satisfiable w.r.t.\ $\mathcal{T}$ iff $A$ is satisfiable w.r.t.\ $\mathcal{T}'$.
\end{prop}
\begin{proof}
	Let $\mathcal{T}$ be an acyclic TBox.
	We apply three steps:
	\begin{itemize}
		\item For each defined concept name $A$ in $\mathcal{T}$, we introduce a fresh name $\overline{A}$.
			Extend $\mathcal{T}$ with concept definitions $\overline{A} \equiv \neg C$ for $A \equiv C \in \mathcal{T}$.
		\item Convert the right-hand sides of all concept definitions into NNF,
			not distinguishing between primitive and defined concepts.
		\item In all concepts definitions, replace every subconcept $\neg A$,
			where $A$ is a defined concept name, with $\overline{A}$.
	\end{itemize}
	The resulting TBox is as required.
\end{proof}

\begin{definition*}[Simple TBox]
	An acyclic TBox $\mathcal{T}$ is simple if all concept definitions are of the from:
	\[
	A \equiv P, A \equiv \neg P, A \equiv B_1 \sqcap B_2, A \equiv B_1 \sqcup b_2, A \equiv \exists r.B_1, A \equiv \forall r.B_1
	\]
	where $P$ is a primitive concept and $B_1, B_2$ are defined concept names.
\end{definition*}
\begin{note}
	Every simple TBox is in NNF.
\end{note}

\begin{lemma}[simple TBox]\label{lem:5.2}
	Let $A_0$ be a concept name.
	There is a polynomial time transformation of each acyclic TBox $\mathcal{T}$
	into a simple TBox $\mathcal{T}'$ such that $A_0$ is satisfiable w.r.t.\ $\mathcal{T}$ 
	iff $A_0$ is satisfiable w.r.t.\ $\mathcal{T}'$.
\end{lemma}
\begin{proof}
	Let $A_0$ be a concept name and $\mathcal{T}$ an acyclic TBox.
	By Proposition \ref{prop:5.1} we can assume that $\mathcal{T}$ is in NNF.
	Apply the following modifications:
	\begin{itemize}
		\item To break down a concept definition $A \equiv C_1 \sqcap C_2$
			with $C_1$ or $C_2$ not a defined concept name,
			we replace $A \equiv C_1 \sqcap C_2$ with $A \equiv B_1 \sqcap B_2$ 
			and $B_1 \equiv C_1$, $B_2 \equiv C_2$.
			Similarly for $A \equiv C_1 \sqcup C_2, A \equiv \exists r.C_1, A \equiv \forall r.C_2$.
		\item Delete each concept definition $A \equiv B$, where $B$ is a defined concept,
			and replace every occurrence of $A$ in $\mathcal{T}$ with $B$ if $A \neq A_0$.
			Otherwise, replace all occurrences of $B$ with $A$.
			($A = B = A_0$ cannot happen due to acyclicity!)
			\qedhere
	\end{itemize}
\end{proof}
So we will only need to test for satisfiability of simple TBoxes.

We will also need the notion of a type:
\begin{definition}[Type]
	Let $\mathcal{T}$ be a simple TBox and $\func{Def}(\mathcal{T})$ the set of defined concept names in $\mathcal{T}$.
	A type for $\mathcal{T}$ is a set $\tau \subseteq \func{Def}(\mathcal{T})$
	that satisfies the following conditions:
	\begin{enumerate}
		\item $A \in \tau$ implies $B \notin \tau$, if $A \equiv P, B \equiv \neg P \in \mathcal{T}$ ;
		\item $A \in \tau$ implies $B \in \tau$ and $B' \in \tau$, if $A \equiv B \sqcap B' \in \mathcal{T}$;
		\item $A \in \tau$ implies $B \in \tau$ or $B' \in \tau$, if $A \equiv B \sqcup B' \in \mathcal{T}$;
	\end{enumerate}
\end{definition}
\begin{note}
	The restriction to defined concepts rather than subconcepts is possible because the TBox is simple.
\end{note}

The satisfiability algorithm tries to construct a tree model
whose depth is bounded by the \textit{role depth} of the input concept name.

\begin{definition*}[role depth]
	Let $A$ be a defined concept name,
	then its role depth $\func{rd}(A)$ is defined as follows:
	\begin{itemize}
		\item If $A \equiv (\neg) P \in \mathcal{T}$, then $\func{rd}(A) = 0$,
		\item If $A \equiv B_1 * B_2 \in \mathcal{T}$ with $* \in \left\{ \sqcap, \sqcup \right\}$,\\
			then $\func{rd}(A) = \func{max}(\func{rd}(B_1), \func{rd}(B_2))$
		\item If $A \equiv Qr.B \in \mathcal{T}$ with $Q \in \left\{ \exists, \forall \right\}$, \\
			then $\func{rd}(A) = \func{rd}(B) + 1$.
	\end{itemize}
\end{definition*}
\begin{note}
	This definition is well-founded since $\mathcal{T}$ is acyclic.
\end{note}
For $i \geq 0$, we define
\[
	\func{Def}_i(\mathcal{T}) \coloneqq \left\{ A \in \func{Def}(\mathcal{T}) \mid \func{rd}(A) \leq i \right\}
.\]

We now have everything at hand for our algorithm.
\begin{mdframed}
	\begin{algorithm}[H]
		\caption{$\mathcal{ALC}\func{-Worlds}(A_0, \mathcal{T})$}
		\label{alg:alc-worlds}
		\begin{algorithmic}[1]
			\Require a concept name $A_0$ and a simple TBox $\mathcal{T}$
			\State $i = \func{rd}(A_0)$
			\State guess a set $\tau \subseteq \func{Def}_i(\mathcal{T})$ with $A_0 \in \tau$ 
			\State $\func{recurse}(\tau, i, \mathcal{T})$
		\end{algorithmic}
	\end{algorithm}
	\begin{algorithm}[H]
		\caption{$\func{recurse}(\tau, i , \mathcal{T})$}
		\label{alg:recurse}
		\begin{algorithmic}[1]
			\If{$\tau$ is not a type for $\mathcal{T}$} \Return false \EndIf
			\If{$i = 0$} \Return true \EndIf
			\ForAll{$A \in \tau$ with $A \equiv \exists r.B \in \mathcal{T}$}
				\State $S = \left\{ B \right\} \cup \left\{ B' \mid \exists A': A' \in \tau \text{ and } A' \equiv \forall r.B' \in \mathcal{T} \right\}$
				\State guess a set $\tau \subseteq \func{Def}_{i-1}(\mathcal{T})$ with $S \subseteq \tau$ 
				\If{$\func{recurse}(\tau, i-1, \mathcal{T})$ returns false} \Return false \EndIf
			\EndFor
			\Return true
		\end{algorithmic}
	\end{algorithm}
\end{mdframed}
