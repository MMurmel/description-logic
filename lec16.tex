\lecture{16}{15.06.2021}{}
A \textit{winning strategy} for Player 1 in $G$ is a finite node-labeled tree $\left( V, E ,l \right)$ of depth $n$, where:
 \begin{itemize}
	 \item $l$ assigns to each node $v \in V$ a configuration $l(v)$ ;
	 \item the root is labeled with the initial configuration $\varepsilon$;
	 \item if $v$ is a node of depth $i < n$ with $i$ even and $l(v) = t$,
		 then $v$ has one successor $v'$ with $l(v')  \in \left\{ t0, t1 \right\}$ ;
	 \item if $v$ is a node of depth $i < n$ with $i$ odd and $l(v) = t$,
		 then $v$ has two successors $v'$ and $v''$ with $l(v') = t0$ and $l(v'') = t1$;
	 \item if $v$ is a node of depth $n$ and $l(v) = t$, then $t$ satisfies $\phi$.
\end{itemize}

We can then construct a $\mathcal{ALC}$ concept $C_G$ for every FBG $G = \left( \phi, \Gamma_1, \Gamma_2 \right)$ such that,
Player 1 has a winning strategy iff $C_G$ is satisfiable.
The general idea is to create $C_G$ is such a way, that the winning strategies are the tree models of $C_G$.
We use one role name $r$ as the edge relation for the tree, and for each propositional variable $p_i$, we define a concept $P_i$.

$C_G$ is a conjunction of the following concept descriptions:
\begin{itemize}
	\item For each node of odd depth $i$, there are two successors, one for each possible truth value of $p_{i+1}$ :
		\[
			C_1 = \bigsqcap_{i \in \left\{ 1,3, \ldots, n-1 \right\}} \forall r^i.(\exists r. \neg P_{i+1} \sqcap \exists r.P_{i+1})
		\]
		where $\forall r^i.C$ denotes the $i$-fold nesting $\forall r. \cdots \forall r.C$.
	\item For each node of even depth $i$, there is one successor:
		\[
			C_2 = \bigsqcap_{i \in \left\{ 0, 2, \ldots, n-2 \right\}} \forall r^i.\exists r. \top
		\]
		i.e.\ the successor is contained in $P_i \sqcup \neg P_i = \top$.
	\item One a truth value is chosen, it remains fixed:
		\[
			C_3 = \bigsqcap_{1 \leq i \leq j <n} \forall r^j. \left( (P_i \implies \forall r.P_i) \sqcap (\neg P_i \implies \forall r. \neg P_i) \right)
		\]
	\item At the leafs, the formula $\phi$ is true:
		\[
			C_4 = \forall r^n. \phi^*
		\]
		where $\phi^*$ denotes the result of converting $\phi$ into an $\mathcal{ALC}$ concept:
		\[
			p_i \to P_i, \land \to \sqcap, \lor \to \sqcup
		\]
\end{itemize}
Thereby, $C_G = C_1 \sqcap \ldots \sqcap C_4$.
Obviously, the size of $C_1, C_2, C_3$ is quadratic in $n$ and the size of $C_4$ is linear in  $n$ plus the size of $\phi$.
Thus, this reduction is actually constructible in polynomial time in the description of $G$.

\begin{lemma}
	Player 1 has a winning strategy in $G$ iff $C_G$ is satisfiable.
\end{lemma}
\begin{proof}
	"$ \implies$":
	Assume that player 1 has a winning strategy $\left( V, E, l \right)$ with root $v_0 \in V$.
	We define the interpretation $\mathcal{I}$ as follows:
	\begin{itemize}
		\item $\Delta^{\mathcal{I}} = V$,
		\item $r^{\mathcal{I}} = E$,
		\item $P_i^{\mathcal{I}} = \left\{ v \in V \mid l(v)  \text{ sets $p_i$ to 1} \right\}$ 
			for $1 \leq i \leq n$.
	\end{itemize}
	It is easy to see that $v_0 \in C_G^\mathcal{I}$.

	"$ \impliedby$":
	Let $\mathcal{I}$ be a model of $C_G$, and let $d_0 \in C_G^\mathcal{I}$.
	We define a winning strategy $\left( V,E,l \right)$ with $V \subseteq \N \times \Delta^\mathcal{I}$.
	The construction will be such that 
	if $(i,d) \in V$, then $d$ is reachable from $d_0$ in $\mathcal{I}$ 
	by traveling  $i$ steps along $r^\mathcal{I}$.
	In the following we call this property (*). 
	Start by setting $V = \left\{ (0,d_0) \right\}$, $E = \emptyset$ and $l(0,d_0) = \varepsilon$.
	We proceed in rounds $1, \ldots, n$.

	In each odd round $i$, iterate over all nodes $(i-1,d) \in V$ and do the following:
	\begin{itemize}
		\item select a $d' \in \Delta^\mathcal{I}$ such that $(d, d') \in r^\mathcal{I}$
			(exists since $d_0 \in C_2^\mathcal{I}$ and (*) by induction)
		\item add $(i, d')$ to  $V$, $\left( (i-1,d),(i,d') \right)$ to $E$
			and set  $l(i,d') = tj$ where  $t = l(i-1,d)$
			and  $j = 1$ if $d' \in P_i^{\mathcal{I}}$
			and $j = 0$ if  $d' \notin P_i^{\mathcal{I}}$.
	\end{itemize}

	In each even round, we iterate over all nodes $(i-1, d) \in V$ and do the following:
	\begin{itemize}
		\item select $d', d'' \in \Delta^\mathcal{I}$ such that $d' \in P_i^\mathcal{I}$,
			$d '' \notin P_i^\mathcal{I}$ and $\left\{ (d,d'), (d,d'') \right\} \subseteq r^\mathcal{I}$ 
			(exists since $d_0 \in C_1^{\mathcal{I}}$ and (*) is satisfied by induction).
		\item add $(i,d')$ and $(i, d'')$ to $V$,
			$\left( (i-1, d), (i,d') \right)$ and $\left( (i-1,d),(i,d'') \right)$ to $E$ 
			and set $l(i,d') = t1$ and $l(i,d'') = t0$
			where $t = l(i-1,d)$.
	\end{itemize}
	Since $d_0 \in C_3^\mathcal{I}$ and $d_0 \in C_4^\mathcal{I}$, it is easy to prove
	that the resulting tree is a winning strategy for Player 1.
\end{proof}

\begin{theorem}
	In $\mathcal{ALC}$, concept satisfiability and subsumption
	without TBoxes and with acyclic TBoxes are \textsc{PSpace}-complete.
\end{theorem}

\subsection{complexity upper bound (case of general TBox)}
When extending the arguments given above to general TBoxes,
it is helpful to only look at a specific structure of concepts and TBoxes.

Without loss of generality we restrict the attention to satisfiability of a concept name
(and not a complex concept) $A_0$ w.r.t.\ a general TBox $\mathcal{T}$ in which this name occurs:
\[
C \text{ is satisfiable w.r.t.\ } \mathcal{T} \iff A_0 \text{ is satisfiable w.r.t.\ }\mathcal{T} \cup \left\{ A_0 \sqsubseteq C \right\}
\]
where $A_0$ is a new name not occurring in $C$ or $\mathcal{T}$.
Furthermore, we can assume that the TBox consists of a single GCI of the form $\top \sqsubseteq C_{\mathcal{T}}$ :
\[
	\begin{split}
		&\mathcal{I} \vDash \left\{ C_1 \sqsubseteq D_1, \ldots, C_n \sqsubseteq D_n \right\} \\ \iff &\mathcal{I} \vDash \left\{ \top \sqsubseteq (\neg C_1 \sqcup D_1) \sqcap \ldots \sqcap (\neg C_n \sqcup D_n) \right\}
	\end{split}
\]
for all interpretations $\mathcal{I}$.
We can also assume that the concept $C_{\mathcal{T}}$ in this GCI is in NNF.

Using the type elimination algorithm explained in the following,
we will prove an \textsc{ExpTime} upper bound for satisfiability w.r.t.\ general $\mathcal{ALC}$ TBoxes.

\begin{definition}[type]
	Let $\mathcal{T}$ be a general TBox satisfying the restrictions described above.
	A type for $\mathcal{T}$ is a set $\tau \subseteq \func{sub}(\mathcal{T})$ satisfying the following conditions:
	\begin{enumerate}[label=(\roman*)]
		\item $A \in \tau \implies \neg A \notin \tau$, for all $\neg A \in \func{sub}(\mathcal{T})$
		\item $C \sqcap D \in \tau \implies C \in \tau \land D \in \tau$, for all $C \sqcap D \in \func{sub}(\mathcal{T})$
		\item $C \sqcup D \in \tau \implies C \in \tau \lor D \in \tau$, for all $C \sqcup D \in \func{sub}(\mathcal{T})$
		\item $C_{\mathcal{T}} \in \tau$
	\end{enumerate}
\end{definition}
\begin{note}
	Since the number of subconcepts of $\mathcal{T}$ is linear in the size of $\mathcal{T}$,
	the number of types is at most exponential in the size of $\mathcal{T}$.
\end{note}

Type elimination then starts with the set of all types and iteratively removes types
whose existential restrictions are not realized by the current set of types.
