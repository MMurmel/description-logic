\lecture{9}{11.05.2021}{}
Before we can introduce the whole algorithm, we must introduce the procedure \texttt{exp}, which does the following:
\begin{itemize}
	\item takes as input a normalized $\mathcal{ALC}$-ABox $\mathcal{A}$, a rule $R$ and an assertion or pair of assertions $\alpha$
		such that $R$ is applicable to $\alpha$ in  $\mathcal{A}$
	\item returns a set $\func{\texttt{exp}}(\mathcal{A},R,\alpha)$ containing all  ABoxes that can result from applying $R$ to $\alpha$ in $\mathcal{A}$.
\end{itemize}
\begin{example}
	\[
	\begin{split}
		&\func{\texttt{exp}}(\left\{ a : \neg D, a : C \sqcup D \right\}, \sqcup \text{-rule}, a : C \sqcup D) \\
		&= \left\{ \left\{  a : \neg D, a : C \sqcup D, a : C\right\}, \left\{  a : \neg D, a : C \sqcup D, a : D\right\} \right\}
	\end{split}
	\]
\end{example}

\begin{definition}[deterministic tableau algorithm]
	We now define the algorithm used for actual reasoning (with deterministic machines):
	\begin{algorithm}[H]
		\caption{consistent($\mathcal{A}$)}
		\label{alg:consistent}
		\begin{algorithmic}[1]
			\Require a normalized $\mathcal{ALC}$-ABox $\mathcal{A}$
		\If{$\func{expand}(\mathcal{A}) \neq \emptyset$}
				\State \textbf{return} "consistent"
			\Else{}
				\State \textbf{return} "inconsistent"
			\EndIf
		\end{algorithmic}
	\end{algorithm}
	The subroutine $\func{expand}(\mathcal{A})$ makes use of the procedure \texttt{exp} discussed earlier:
	\begin{algorithm}[H]
		\caption{expand($\mathcal{A}$)}
		\label{alg:expand}
		\begin{algorithmic}[1]
			\Require a normalized $\mathcal{ALC}$-ABox $\mathcal{A}$ 
			\If{$\mathcal{A}$ is not complete}
				\State select a rule $R$ that is applicable to $\mathcal{A}$ and an assertion or 
				\State a pair of assertions $\alpha$ in $\mathcal{A}$ to which $R$ is applicable
				\If{$\exists \mathcal{A}' \in \func{\texttt{exp}}(\mathcal{A},R,\alpha)$ with $\func{expand}(\mathcal{A}') \neq \emptyset$}
					\State \textbf{return} $\func{expand}(\mathcal{A}')$
				\Else
						\State \textbf{return} $\emptyset$
				\EndIf
			\Else
				\If{$\mathcal{A}$ contains a clash}
					\State \textbf{return} $\emptyset$
				\Else
					\State \textbf{return} $\mathcal{A}$
				\EndIf
			\EndIf
		\end{algorithmic}
	\end{algorithm}
\end{definition}
To show that this decision procedure is correct, we will need to show:
\begin{enumerate}
	\item Termination
	\item Completeness
	\item Soundness
\end{enumerate}

To show termination some auxiliary definitions and results are needed.
We expand the definition of subconcepts to ABoxes:
\[
	\func{sub}(\mathcal{A}) = \bigcup_{a:C \in \mathcal{A}} \func{sub}(C)
.\]
And to knowledge bases $\mathcal{K} = (\mathcal{T}, \mathcal{A})$:
\[
	\func{sub}(\mathcal{K}) = \func{sub}(\mathcal{T}) \cup \func{sub}(\mathcal{A})
.\]
Furthermore, we define the set of concepts occurring in a concept assertion as follows:
\[
	\func{con}_{\mathcal{A}}(a) = \left\{ C \mid a : C \in \mathcal{A} \right\}
.\]

\begin{lemma}\label{lem:cardinality bound abox subconcepts}
	For each $\mathcal{ALC}$-ABox $\mathcal{A}$, we have:
	\[
		\lvert \func{sub}(\mathcal{A}) \rvert \leq \sum_{a:C \in \mathcal{A}} \func{size}(C) 
	.\]
\end{lemma}

\begin{lemma}[Termination]
	For each normalized $\mathcal{ALC}$-ABox $\mathcal{A}$, $\func{consistent}(\mathcal{A})$ terminates.
\end{lemma}
\begin{proof}
	Let $m = \lvert \func{sub}(\mathcal{A}) \rvert$.
	Termination follows from the following properties of the expansion rules:
	\begin{itemize}
		\item The expansion rules never remove an assertion from $\mathcal{A}$,
			and each rule application adds a new assertion of the form $a:C$ for $C \in \func{sub}(\mathcal{A})$.
			Moreover, we saw in lemma \ref{lem:card(sub)<=size} and lemma \ref{lem:cardinality bound abox subconcepts} that the size of $\func{sub}(\mathcal{A})$ is bounded by the size of $\mathcal{A}$.
			Thus, for a given individual $a$ we can have at most $m$ rule applications that add a concept assertion for $a$,
			and $\lvert \func{con}_{\mathcal{A}}(a) \rvert \leq m$.
		\item A new individual name is added to $\mathcal{A}$ only when an existential rule is applied to an assertion $a:C$ with $C$ is an existential restriction.
			Each such assertion triggers the $\exists$-rule only once.
			Thus, a given individual name can trigger the application of the $\exists$-rule at most $m$ times.
			This means each individual can have at most $m$ (newly added) successors.
		\item The $\exists$- and $\forall$-rules are triggered by assertions of the form $a : \exists r.C$ and $a : \forall r.C$, and they only add concept assertions $b:C$, where $b$ is a successor of $a$.
			Thus, for every concept assertion $b:C$ in a tree individual (see script), its predecessor has a concept assertion for a larger concept.
			Thus, when we go from an individual to its successor in the tree, the maximum size of concept assertions decreases.
			Since we start at the root individuals with a finite maximum size, there cannot be infinite paths in the tree. \qedhere
	\end{itemize}
\end{proof}

\begin{lemma}[Soundness]
	If $\func{consistent}(\mathcal{A})$ returns "consistent", then $\mathcal{A}$ is consistent.
\end{lemma}
\begin{proof}
	Let $\mathcal{A}'$ be the set returned by $\func{expand}(\mathcal{A})$.
	Since the algorithm returns "consistent", $\mathcal{A}'$ is a complete and clash-free ABox.
	We use $\mathcal{A}'$ to define an interpretation $\mathcal{I}$ and show that it is a model of $\mathcal{A}'$.
	\begin{align*}
		\Delta^\mathcal{I} &= \left\{ a \mid a:C \in \mathcal{A}' \right\}\\
		a^\mathcal{I} &= a \text{ for each individual name $a$ occurring in $\mathcal{A}'$} \\
		A^\mathcal{I} &= \left\{ a \mid A \in \func{con}_{\mathcal{A}'}(a) \right\} \text{ for each concept name } A \in \func{sub}(\mathcal{A'}) \\
		r^\mathcal{I} &= \left\{ (a,b) \mid (a,b):r \in \mathcal{A}' \right\} \text{ for each role $r$ occurring in $\mathcal{A}'$}
	\end{align*}
	Since the expansion roles never delete assertions, we have $\mathcal{A} \subseteq \mathcal{A}'$, so $\mathcal{I}$ is also a model of $\mathcal{A}$.
	It remains to show that $\mathcal{I}$ is indeed a model of $\mathcal{A}'$.
	\begin{itemize}
		\item role assertions: For every $(a,b) : r \in \mathcal{A}'$ we have $(a^\mathcal{I},b^\mathcal{I}) = (a,b) \in r^\mathcal{I}$ by the definition of $\mathcal{I}$.
		\item concept assertions: For all $a : C \in \mathcal{A}'$ we show that $a^\mathcal{I} = a \in C^\mathcal{I}$ by induction on the structure of $C$:
		% proof in video of lecture 10, but fits here
			Induction basis: $C$ is a concept name.
			By definition of $\mathcal{I}$, if $a : C \in \mathcal{A}'$, then $a = a^\mathcal{I} \in C^\mathcal{I}$.

			Induction step: For every constructor:
			\begin{itemize}
				\item $C = \neg D$ : Since $\mathcal{A}'$ is clash free, $a : \neg D \in \mathcal{A}'$ implies that $a : D \notin \mathcal{A}'$.
					Since all concepts in $\mathcal{A}$ are in NNF, $D$ must be a concept name.
					By definition of $\mathcal{I}$, $a^\mathcal{I} \notin D^\mathcal{I}$,
					which implies $a^\mathcal{I} = a \in \Delta^\mathcal{I} \setminus D^\mathcal{I} = (\neg D)^\mathcal{I} = C^\mathcal{I}$.
				\item $C = D \sqcup E$ : If $a : D \sqcup E \in \mathcal{A}'$, then completeness of $\mathcal{A}'$ implies that
					$\left\{ a:D, a:E \right\} \cap \mathcal{A}' \neq \emptyset$, since otherwise the $\sqcup$-rule would be applicable.
					By induction, this implies that $a^\mathcal{I} \in D^\mathcal{I}$ or $a^\mathcal{I} \in E^\mathcal{I}$.
					In both cases $a^\mathcal{I} \in C^\mathcal{I} \cup E^\mathcal{I} = (D \sqcup E)^\mathcal{I} = C^\mathcal{I}$.
				\item $C = D \sqcap E$: can be treated similarly.
				\item $C = \forall r.D$ : Let $a : \forall r.D \in \mathcal{A}'$.
					To show that $a^\mathcal{I} \in (\forall r.D)^\mathcal{I}$, we consider $b^\mathcal{I}$ such that $(a^\mathcal{I}, b^\mathcal{I}) \in r^\mathcal{I}$.
					We must show that $b^\mathcal{I} \in D^\mathcal{I}$.
					Now $(a^\mathcal{I}, b^\mathcal{I}) \in r^\mathcal{I}$ implies $(a,b) : r \in \mathcal{A}'$.
					Since $\mathcal{A}'$ is complete, the $\forall$-rule is not applicable, which implies $b : D \in \mathcal{A}'$.
					Induction yields $b^\mathcal{I} \in D^\mathcal{I}$, which shows $a^\mathcal{I} \in (\forall r.D)^\mathcal{I} = C^\mathcal{I}$.
				\item $C = \exists r.D$: can be treated similarly.
					\qedhere
			\end{itemize}
	\end{itemize}
\end{proof}
