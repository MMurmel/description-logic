\subsubsection{Termination \& \textsc{PSpace}}
\lecture{15}{10.06.2021}{}
Firstly, we will show, that this algorithm terminates and always runs in \textsc{PSpace}.
For this, we will consider the recursion tree corresponding to the recursive calls of $\func{recurse}$:

\begin{definition*}[recursion tree]
	A \textit{recursion tree} is a tuple $T = (V, E, l)$ such that:
	\begin{itemize}
		\item $(V,E)$ is a tree whose nodes correspond to the calls of $\func{recurse}$;
		\item $l$ is a node labeling function that assigns with each node $v \in V$ 
			the arguments $l(v) = (\tau, i, \mathcal{T})$ of the recursive call corresponding to $v$;
		\item $(v, v') \in E$ if the call corresponding to $v'$ occurred dur
	\end{itemize}
\end{definition*}

The depth of the recursion tree is bounded by $\func{rd}(A_0) \leq \func{size}(\mathcal{T})$,
its outdegree is bounded by the number of concept definitions in $\mathcal{T}$.
Therefore this tree must be finite, which ensures termination of the algorithm.

The number of entries in the recursion stack is bounded by the depth of the recursion tree,
therefore, only polynomially many (even linearly many) entries of the recursion stack
must be held in memory at any time.
Furthermore, the size of each entry in the recursion stack is bounded by the size of $\mathcal{T}$.
All in all, this yields, that  $\mathcal{ALC}\func{-Worlds}$ is in \textsc{NPSpace} (due to the guessing),
which, by Savitch's theorem, means that it is also in  \textsc{PSpace}.

\subsubsection{Soundness and Completeness}
\medskip
\begin{lemma}\label{lem:5.4}
	$\mathcal{ALC}\func{-Worlds}(A_0, \mathcal{T}) = \var{true}	\iff A_0$ is satisfiable w.r.t.\ $\mathcal{T}$
\end{lemma}
\begin{proof}
	"$ \implies$":
	Let $T = (V, E, l)$ be the recursion tree of a successful run of $\mathcal{ALC}\func{-Worlds}$
	on $A_0$ and $\mathcal{T}$, and root $v_0 \in V$.
	For each node $v \in V \setminus \left\{ v_0 \right\}$,
	let $\sigma(v)$ be the role name that the for all loop was processing when making the call $v$.
	Let $\Delta^\mathcal{I} = V$ and define, for each primitive concept name $P$ and role name $r$ :
	\begin{itemize}
		\item $P^{\mathcal{I}} \coloneqq \left\{ v \in \Delta^\mathcal{I} \mid \exists A: A \in \tau \text{ for } l(v) = (\tau, \cdot, \cdot) \text{ and } A \equiv P \in \mathcal{T} \right\}$
		\item $r^\mathcal{I} \coloneqq \left\{ (v,v') \mid (v,v') \in E \land \sigma(v') = r \right\}$
	\end{itemize}
	For $A, B \in \func{Def}(\mathcal{T})$ we set $A \prec B$ if
	$A \equiv C \in \mathcal{T}$ and $B$ is a subconcept of $C$.
	Let $\prec^+$ be the transitive closure of $\prec$.
	Then, $\prec^+$ is irreflexive, and thus a strict order.
	We define the interpretation of defined concepts by induction on $\prec^+$:
	If  $A \equiv C \in \mathcal{T}$, then $A^\mathcal{I} = C^\mathcal{I}$,
	where $C^\mathcal{I}$ contains only defined concepts smaller than $A$, i.e.\ is well defined.
	
	By its definition, $\mathcal{I}$ is a model of $\mathcal{T}$.
	We want to show that $A_0 \neq \emptyset$ by proving that $v_0 \in A_0^\mathcal{I}$.
	To this end, we proof the following by induction on $\prec^+$:
	Claim: For all  $A \in \func{Def}(\mathcal{T})$ and all $v \in V$ :
	\[
		A \in \tau \text{ for } l(v) = (\tau, \cdot, \cdot) \implies v \in A^{\mathcal{I}}
	.\]
	We only show this for the case where $A \equiv \neg P$ as an example.
	Let $A \in \tau \text{ with } l(v) = (\tau, \cdot, \cdot)$.
	By property 1) of types, there is no $B \in \tau$ with $B \equiv P \in \mathcal{T}$.
	By definition of $\mathcal{I}$, this implies $v \notin P^\mathcal{I}$, as required.
	Thus, $v \in (\neg P)^\mathcal{I} = A^\mathcal{I}$.
	\newline
	"$ \impliedby$":
	Assume $A_0$ is satisfiable w.r.t.\ $\mathcal{T}$.
	Let $\mathcal{I}$ be a model of $\mathcal{T}$ and $d_0 \in \Delta^\mathcal{I}$
	such that $d_0 \in A_0^\mathcal{I}$.
	For $d \in \Delta^\mathcal{I}$ and $i \leq \func{rd}(A_0)$ we set:
	\[
		\func{tp}_i(d) \coloneqq \left\{ A \in \func{Def}_i(\mathcal{T}) \mid d \in A^\mathcal{I} \right\}
	.\]
	We use $\mathcal{I}$ to guide the non-deterministic choices in a run of the algorithm.
	To do this, we pass an element $d \in \Delta^\mathcal{I}$ as a virtual fourth argument
	to the procedure $\func{recurse}$ such that $d \in A^\mathcal{I}$ for all $A$ in first argument $\tau$.
	Initially, we guide the algorithm to guess $\func{tp}_{\func{rd}(A_0)}(d_0)$ as the set $\tau$.

	Now let $\func{recurse}$ be called with arguments $\left( \tau, i, \mathcal{T}, d \right)$
	and assume that the for loop is processing $A \in \tau$ with $A \equiv \exists r.B \in \mathcal{T}$.
	Then $d \in A^\mathcal{I} = (\exists r.B)^\mathcal{I}$ and thus there is a $d' \in B^\mathcal{I}$ with $(d,d') \in r^\mathcal{I}$.
	We guide the algorithm to guess $\func{tp}_{i-1}(d')$ as the type in the recursive call
	and use $d'$ as fourth argument for that call.

	It remains to show that, guiding it in this way, the algorithm returns $\var{true}$.
	This means, all the guessed sets need to be types,
	which is clear since they are induced by individuals in $\mathcal{I}$ and $\mathcal{I}$ is a model of $\mathcal{T}$.
\end{proof}

\begin{theorem}
	In $\mathcal{ALC}$, concept satisfiability and subsumption w.r.t.\ acyclic TBoxes is in \textsc{PSpace}.
\end{theorem}

\subsection{complexity lower bound (case without TBox)}
In order to show  \textsc{PSpace}-hardness for $\mathcal{ALC}$-satisfiability,
we will reduce an other \textsc{PSpace}-hard problem to it.

While the original proof of \textsc{PSpace}-hardness used a reduction from \textsc{QBF-Sat},
we use a game played on formulae of propositional logic, called \textit{finite Boolean game} (FBG).

\begin{definition*}[FBG]
	A \textit{finite Boolean game} (FBG) is a triple $(\phi, \Gamma_1, \Gamma_2)$ with:
	\begin{itemize}
		\item $\phi$ a formula of propositional logic,
		\item $\Gamma_1 \uplus \Gamma_2$ a partition of the variables used in $\phi$ into two sets of identical cardinality.
	\end{itemize}
	The game is played by both players alternating turns in which they determine the truth values of one variable:
	\begin{itemize}
		\item Player 1 controls the variables in $\Gamma_1$ and tries to make $\phi$ true,
		\item Player 2 controls the variables in $\Gamma_2$ and tries to make $\phi$ false.
	\end{itemize}
\end{definition*}

In the FBG-Problem, one wants to decide, whether Player 1 has a winning strategy,
i.e.\ can force a win no matter what Player 2 does.
Lets see how these winning strategies would look like:

Considering any FBG $G = (\phi, \Gamma_1, \Gamma_2)$, let $n = \lvert \Gamma_1 \uplus \Gamma_2 \rvert$,
$\Gamma_1 = \left\{ p_1, p_3, \ldots, p_{n-1} \right\}$ and $\Gamma_2 = \left\{ p_2, p_4, \ldots, p_n \right\}$.
We define a \textit{configuration} of $G$ as a word $t \in \left\{ 0,1 \right\}^i$, for some $i \leq n$:
\begin{itemize}
	\item there were already $i$ steps played in the game;
	\item the $k$th letter of $t$ is the truth value chosen for $p_k$.
\end{itemize}
Obviously, the initial configuration of $G$ is the empty word $\varepsilon$.
A Move is played as follows:
If the current configuration is  $t$, then a truth value for $p_{\lvert t \rvert + 1}$ is selected
\begin{itemize}
	\item by Player 1 if $\lvert t \rvert$ is even,
	\item by Player 2 if $\lvert t \rvert$ is odd.
\end{itemize}
