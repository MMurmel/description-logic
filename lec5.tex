\lecture{5}{27.04.2021}{}
So as long as we are not interested in the efficiency of reasoning, we do not need to develop reasoning procedures for TBoxes.
However, this reduction is in general not polynomial, since the expanded versions may be exponential in the size of $\mathcal{T}$.
One can easily see this with the following example on a TBox:
\begin{align*}
	A_0 &\equiv \forall r.A_1 \sqcap \forall s. A_1 \\
	A_1 &\equiv \forall r.A_2 \sqcap \forall s.A_2 \\
	\vdots & \\
	A_{n-1} & \equiv \forall r.A_n \sqcap \forall s.A_{n}
\end{align*}
The size of this TBox is linear in $n$,
but the expansion version $\widehat{A}_0$ of $A_0$ contains $A_n$ $2^n$ times.

\subsection{Relationship with FOL}
Like everything so far, reasoning in $\mathcal{ALC}$ can be translated into reasoning in FOL:
\begin{lemma}
	Let $\mathcal{K} = ( \mathcal{T},\mathcal{A})$ be a knowledge base, $C,D$ concept descriptions and $a$ an individual name.
	\begin{enumerate}
		\item $C \sqsubseteq_{\mathcal{T}} D \iff \tau(\mathcal{T}) \vDash \forall x.(\tau_x(C)(x) \Rightarrow \tau_x(D)(x))$
		\item $ \mathcal{K}$ is consistent $ \iff \tau(\mathcal{K})$ is consistent
		\item $a$ is an instance of $C$ w.r.t.\ $\mathcal{K} \iff \tau \left( \mathcal{K} \right) \vDash \tau_x(C)(a)$
	\end{enumerate}
\end{lemma}

\begin{proof}
	Exercise!	
\end{proof}

\newpage
\section{Further problems}
So far we have seen the basics of the $\mathcal{ALC}$-DL system and its inference problems.
Often times it is not sufficient to just know \textit{if} one concept is subsumed by another,
but which concepts are actually subsumed by which.
\textbf{Classification} is the task to compute the subsumption hierarchy of \textit{all} concept names occurring in the TBox.
\newline
Likewise, it could be of interest to know, which is the most specific concept name in the TBox to which an ABox individual belongs.
This problem is called \textbf{Realization}.
\newline
Therefore DL-systems often include solution procedures for these problems.

\section{Further relationships with FOL}
We will now introduce a new constructor for illustrative purposes.
If $r$ is a role name and $C$ is a concept description, then $\exists r^+.C $
is a concept description, with the semantics
\[
\begin{split}
	(\exists r^+.C)^\mathcal{I} \coloneqq \left\{ d \in \Delta^{\mathcal{I}} \mid \exists n \geq 1, \exists d_1,\cdots, d_n \in \Delta^\mathcal{I} : \right.\\
		\left. (d,d_1) \in r^\mathcal{I}, \cdots, (d_{n-1},dn) \in r^\mathcal{I} \land d_n \in C^\mathcal{I}\right\}
\end{split}
\]
We claim, that $\exists r^+.C$ cannot be expressed in $\mathcal{ALC}$.
\begin{proof}
	Assume that there is an $\mathcal{ALC}$ concept $D$ such that $D \equiv \exists r^+.C$.
	Consider the ABox $\mathcal{A} = \left\{ a:D \right\}$.
	Let $ \mathcal{B} = \left\{ a : \forall r.\neg C, a: \forall r.\forall r.\neg C, a:\forall r.\forall r.\forall r.\neg C,\ldots \right\}$.
	Then $\mathcal{A} \cup \mathcal{B}$ is inconsistent.
	Since $\mathcal{A} \cup \mathcal{B}$ can be translated to FOL, compactness applies, i.e.,
	there is a finite subset $\mathcal{C} \subseteq \mathcal{A} \cup \mathcal{B}$ such that $\mathcal{C}$ is inconsistent.
	Let $n$ be the maximal nesting of value-restrictions in $\mathcal{C}$, i.e.,
	$\mathcal{C}$ does not contain $\underbrace{\forall r. \forall r. \ldots \forall r.}_{n+1} \neg C$.
	But then there obviously is a model of $\mathcal{C}$. $\contra$
\end{proof}
This shows, that certain concepts can be expressed in FOL, but not in $\mathcal{ALC}$.

\chapter{A Little Bit of Model Theory}
Like we already saw, interpretations of $\mathcal{ALC}$ can be viewed as graphs.
We introduce the notion of bisimulation between graphs and therefore between interpretations.
We will then show, that $\mathcal{ALC}$-concepts cannot distinguish bisimular nodes.
Afterwards these results will grant us the power to show restrictions of the expressive power of $\mathcal{ALC}$ and
interesting properties of models of $\mathcal{ALC}$:
\begin{itemize}
	\item tree model property
	\item closure under disjoint union
\end{itemize}
Furthermore we will show the very important property of $\mathcal{ALC}$: the finite model theory.

\section{Bisimulation}
\begin{definition}[Bisimulation]\label{def:bisimulation}
	Let $\mathcal{I}_1$ and $\mathcal{I}_2$ be interpretations.
	The relation $\rho \subseteq \Delta^{\mathcal{I}_{1}} \times \Delta^{\mathcal{I}_2}$ is a bisimulation between $\mathcal{I}_1$ and $\mathcal{I}_2$ iff:
	\begin{enumerate}[label=(\roman*)]
		\item $d_1 \rho d_2 \implies d_1 \in A^{\mathcal{I}_1} \text{ iff } d_2 \in A^{\mathcal{I}_2}$ for all $A \in \mathscr{C}$
		\item $d_1 \rho d_2 \land  (d_1, d_1') \in r^{\mathcal{I}_1}$ implies the existence of $d_2' \in \Delta^{\mathcal{I}_2}$ such that
			$d_1' \rho d_2' \land (d_2, d_2') \in r^{\mathcal{I}_2}$ for all $r \in \mathscr{R}$
		\item $d_1 \rho d_2 \land  (d_2, d_2') \in r^{\mathcal{I}_2}$ implies the existence of $d_1' \in \Delta^{\mathcal{I}_1}$ such that
			$d_1' \rho d_2' \land (d_1, d_1') \in r^{\mathcal{I}_1}$ for all $r \in \mathscr{R}$
	\end{enumerate}
\end{definition}

\begin{note}
	\begin{itemize}
		\item $\mathcal{I}_1 = \mathcal{I}_2$ is possible
		\item the empty relation $\emptyset$ is always a bisimulation.
	\end{itemize}
\end{note}

\begin{notation}
	Let $\mathcal{I}_1, \mathcal{I}_2$ be interpretations and $d_1 \in \Delta^{\mathcal{I}_1}, d_2 \in \Delta^{\mathcal{I}_2}$.
	We then write:
	\[
		\left( \mathcal{I}_1, d_1 \right) \sim \left( \mathcal{I}_2, d_2 \right) \iff \exists \text{ a bisimulation $\rho$ between  $\mathcal{I}_1, \mathcal{I}_2$
		such that $d_1 \rho d_2$}
	.\]
	And we say "$d_1$ in $\mathcal{I}_1$ is bisimular to $d_2$ in $\mathcal{I}_2$".
\end{notation}

\begin{theorem}[Bisimulation invariance of $\mathcal{ALC}$]\label{thm: bisimulation invariance}
	If $ \left( \mathcal{I}_1,d_1 \right) \sim \left( \mathcal{I}_2, d_2 \right)$, then the following holds for all $\mathcal{ALC}$-concepts $C$ :
	\[
		d_1 \in C^{\mathcal{I}_1} \iff d_2 \in C^{\mathcal{I}_2}
	.\]
\end{theorem}
This theorem essentially tells us, that $\mathcal{ALC}$-concepts cannot distinguish between bisimular elements.
