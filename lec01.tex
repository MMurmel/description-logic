% Chapter counting starts at 0
\setcounter{chapter}{-1}
\chapter{Preface}
This is my transcript of the lecture "Description Logic" held by Prof. Dr. Franz Baader at TU Dresden in the summer semester 2021.
The lecture is based on the book "An Introduction to Description Logic" \cite{baader_dl_17} co-authored by Prof. Baader.

\chapter{Introduction}
\lecture{1}{18.04.2021}{}
Description Logic (DL) is a subfield of knowledge representation which intern is a subfield of Artificial Intelligence.
Note, that there is not \textbf{one} DL, but many different ones, most of which can be seen as fragments of FOL.
The general goal of Knowledge Representation and therefore of DL can be summarized as follows:
\begin{itemize}
	\item formalism: well-defined syntax and unambiguous semantics
	\item high-level description: only relevant aspects need to be represented
	\item intelligent applications: ability to infer implicit knowledge from the explicitly represented one
	\item effective: practical for reasoning tools and efficient implementable
\end{itemize}
Especially the latter point is interesting, since FOL is undecidable.
Therefore we must strike the right balance when it comes to expressive power.
Too low and not all the relevant knowledge can be presented, too high and reasoning becomes inefficient (or impossible).

The reasoning procedures used for processing the knowledge represented by a DL should satisfy the following properties:
\begin{itemize}
	\item soundness: Positive answers are indeed correct. 
	\item correctness: All positive answers are found, e.g.\ negative answers are correct.
	\item termination: There will always be an answer in finite time.
\end{itemize}
The procedure should furthermore be as efficient and as practical as possible.

To automatically reason about implicitly represented knowledge in a DL, one has to formalize the following:
\begin{itemize}
	\item syntax and semantics used in the DL (\textbf{description language}),
	\item the terminological knowledge of the application domain (\textbf{TBox}),
	\item the facts known about the specific "world" instance (\textbf{ABox}).
\end{itemize}

\chapter{A basic Description Logic}
In this chapter we will have a look at a basic DL called $\mathcal{ALC}$. 
The names of DLs follow a common naming scheme:
\begin{itemize}
	\item the basic language is called "attributive language"
	\item additional constructors are abbreviated with letters that are added after $ \mathcal{AL} $ 
	\item $ \mathcal{C} $ stands for \textit{complement}, i.e., $ \mathcal{ALC} $ is obtained from $ \mathcal{AL} $ by adding the complement ($ \neg $) operator
\end{itemize}

\section{The description language of $\mathcal{ALC}$}
\subsection{Syntax and semantics of $\mathcal{ALC}$}
\begin{definition}[Syntax of $\mathcal{ALC}$]
	Let $\mathscr{C}$ and $\mathscr{R}$ be disjoint sets of concept names and role names, respectively.
	$\mathcal{ALC}$-concept descriptions are inductively defined.
	\begin{itemize}
		\item If $ A \in \mathscr{C} $, then $ A $ is an $ \mathcal{ALC}$-concept description.
		\item If $ C,D $ are $ \mathcal{ALC}$-concept description, and $ r \in \mathscr{R} $, then the following are $\mathcal{ALC}$-concept descriptions:
			\begin{itemize}
				\item $ C \sqcap D $ (conjunction)
				\item $ C \sqcup D $ (disjunction)
				\item $ \neg C $ (negation)
				\item $ \forall r.C $ (value restriction)
				\item $ \exists r.C $ (existential restriction)
			\end{itemize}
	\end{itemize}
\end{definition}

\begin{note}
	One could equivalently define $\bot$ and $\top$ as concept descriptions.
	The main differences lie in the structure of induction proves and the length of formulas, that may be used in complexity analysis.
\end{note}

\begin{notation}
	We allow ourselves the use of parentheses for better structure and to define the following abbreviations:
	\begin{itemize}
		\item $ \top \coloneqq A \sqcup \neg A $ (top)
		\item $ \bot \coloneqq A \sqcap \neg A $ (bottom)
		\item $ C \Rightarrow D \coloneqq \neg C \sqcup D $ (implication)
	\end{itemize}
	Furthermore, we will use these conventions:
	\begin{itemize}
		\item Concept names are called \textit{atomic}.
		\item All other $\mathcal{ALC}$-concept descriptions are called \textit{complex}.
		\item We often abbreviate \textit{$\mathcal{ALC}$-concept description} with \textit{concept description} or just \textit{concept}.
		\item We often use $A, B$ for concept names, $C, D$ for (possibly) complex concepts and $r, s$ for role names.
		\item When talking about concrete elements of $\mathscr{C}$ or $\mathscr{R}$ we will use capitalized concept names and lowercased role names.
	\end{itemize}
\end{notation}

\begin{example}
	Here are some examples of the $\mathcal{ALC}$-syntax.
	Although the naming is chosen to have intuitive meaning, we have not defined a formal semantics of this description language yet.
	\begin{itemize}
		\item $ \var{Person} \sqcap \var{Female}$
		\item $ \var{Student} \sqcap \forall \var{attends}.(\var{Lecture}\sqcap \neg \var{Boring})$
	\end{itemize}
\end{example}
