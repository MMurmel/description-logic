\lecture{22}{08.07.2021}{}
We use the following classification rules:
\begin{mdframed}[frametitle=Classification rules for $\mathcal{ELI}$]
	\begin{align*}
		&\infer[\text{i.CR1}]{ K \sqsubseteq \left\{ A \right\}}{}\qquad\text{if $A \in K$ and $K$ occurs in $\mathcal{T}'$}\\\\
		&\infer[\text{i.CR2}]{M \sqsubseteq C}{M \sqsubseteq \left\{ B \right\} \text{ for all } B \in K & K \sqsubseteq C}\quad\text{if $M$ occurs in $\mathcal{T}'$}\\\\
		&\infer[\text{i.CR3}]{M_2 \sqsubseteq \left\{ A \right\}}{M_2 \sqsubseteq \exists r.M_1 & M_1 \sqsubseteq \forall r^-.\left\{ A \right\}}\\\\
		&\infer[\text{i.CR4}]{M_1 \sqsubseteq \exists r.(M_2 \cup \left\{ A \right\})}{M_1 \sqsubseteq \exists r.M_2 & M_1 \sqsubseteq \forall r.\left\{ A \right\}}
	\end{align*}
\end{mdframed}
\begin{note}
	Again, these are not concrete rules, but rule schemata.
	Concrete instances are are obtained by replacing
	\begin{itemize}
		\item $K, M, M_1, M_2$ by sets of concept names in $\func{sig}(\mathcal{T})$,
		\item $A, B$ by a concept name in $\func{sig}(\mathcal{T})$,
		\item $r$ by a role name or inverse of a role name in $\func{sig}(\mathcal{T})$,
		\item $C$ by any admissible right-hand side of a $\mathcal{T}$-i.sequent.
	\end{itemize}
\end{note}
\begin{note}
	In i.CR1, only instantiations are allowed for which $K$ actually occurs explicitly in some $\mathcal{T}$-i.sequent
	in the current TBox $\mathcal{T}'$.
	By this, the worst case still runs in exponential time, but in the best case, the algorithm is better than exponential.
	The analogous restriction on $M$ in rule i.CR2 is needed in the case where $K = \emptyset$.
	In contrast, rule i.CR4 can generate new sets, and thus may cause an exponential blowup.
\end{note}
\begin{example}[Exponential blowup]
	Consider the following TBox:
	\[
	\mathcal{T} \coloneqq \left\{ A \sqsubseteq \exists r.\top \right\} \cup \left\{ \exists r^-.A \sqsubseteq A_i \mid i= 1, \ldots, n\right\}
	.\]
	After i.normalization we obtain:
	\[
		\mathcal{T}' \coloneqq \left\{ \left\{ A \right\} \sqsubseteq \exists r.\emptyset \right\} \cup \left\{ \left\{ A \right\} \sqsubseteq \forall r.\left\{ A_i \right\} \mid i = 1, \ldots, n \right\}
	.\]
	It is easy to see that i.CR4 can be used to generate
	\[
	\left\{ A \right\} \sqsubseteq \exists r.M \text{ for all } M \subseteq \left\{ A_1, \ldots, A_n \right\}
	.\]
	Thus, if we add $M \sqsubseteq \forall r^-.\left\{ B \right\}$ to $\mathcal{T}'$,
	we can derive $\left\{ A \right\} \sqsubseteq \left\{ B \right\}$ by applying i.CR3.
	We could also just derive $\left\{ A \right\} \sqsubseteq \exists r.\left\{ A_1, \ldots, A_n \right\}$
	by $n$ applications of rule i.CR4,
	and then derive from this $\left\{ A \right\} \sqsubseteq \left\{ B \right\}$
	similarly to what we did in another example shown in the lecture.
	We can show, however, that there are examples, where no strategy for rule applications can avoid the exponential blowup.
\end{example}
The classification algorithm for $\mathcal{ELI}$ starts by i.Saturation of $\mathcal{T}$:
\begin{itemize}
	\item apply the classification rules exhaustively to the input TBox $\mathcal{T}$ 
	\item the resulting TBox $\mathcal{T}^*$ is called the i.saturated TBox
\end{itemize}
The i.saturated TBox $\mathcal{T}^*$ is again uniquely determined by $\mathcal{T}$.

\addtocounter{definition}{3}
\begin{prop}[soundness and completeness]
	For all concept names $A, B \in \func{sig}(\mathcal{T})$ such that $ \left\{ A \right\}$ occurs in $\mathcal{T}^*$ we have
	\[
	\mathcal{T} \vDash A \sqsubseteq B \iff \left\{ A \right\} \sqsubseteq \left\{ B \right\} \in \mathcal{T}^*.
	.\]
	The condition that $ \left\{ A \right\}$ occurs in $\mathcal{T}^*$ can easily be satisfied by adding $A \sqsubseteq A$ to the input TBox.
	Thus, we get the $\mathcal{T}$-i.sequent $\left\{ A \right\} \sqsubseteq \left\{ A \right\}$.
\end{prop}
Soundness is an easy consequence of the next lemma and the fact that any GCI in $\mathcal{T}$ follows from $\mathcal{T}$.
\begin{lemma}[soundness]
	Assume that
	\begin{itemize}
		\item all the GCIs in $\mathcal{T}'$ follow from $\mathcal{T}$ and
		\item the $\mathcal{T}$-i.sequent above the line of one of the classification rules belong to $\mathcal{T}'$.
	\end{itemize}
	Then the $\mathcal{T}$-i.sequent below the line also follows from $\mathcal{T}$.
\end{lemma}
\begin{proof}
	\begin{itemize}
		\item i.CR1: follows from the fact that a conjunction of concept names is subsumed by each of its conjuncts.
		\item i.CR2: is due to transitivity of subsumption the fact that $E \sqsubseteq F_1$ and $E \sqsubseteq F_2$ implies $E \sqsubseteq F_1 \sqcap F_2$.
		\item i.CR3: note that $\mathcal{T} \vDash M_1 \sqsubseteq \forall r^-.\left\{ A \right\} \iff \mathcal{T} \vDash \exists r.M_1 \sqsubseteq \left\{ A \right\}$.
			Thus, transitivity applies.
		\item i.CR4: Consider a model $\mathcal{I}$ of the GCIs above the line.
			By the GCIs above the line, we know that any element of $M_1^{\mathcal{I}}$ has an
			$r^{\mathcal{I}}$-successor that belongs to $M_2^{\mathcal{I}}$ (first GCI),
			but this $r^{\mathcal{I}}$-successor must also belong to $A^{\mathcal{I}}$ (second GCI),
			and thus to $\left( M \cup \left\{ A \right\} \right)^{\mathcal{I}}$.
			This shows that element of $M_1^{\mathcal{I}}$ also belongs to $\left( \exists r.\left\{ M_2 \cup \left\{ A \right\} \right\} \right)^{\mathcal{I}}$.
			\qedhere
	\end{itemize}
\end{proof}
To show completeness we construct an appropriate canonical interpretation from the i.saturated TBox.
\begin{definition}[canonical interpretation]
	Let $\mathcal{T}$ be a general $\mathcal{ELI}$ TBox in i.normal form and $\mathcal{T}^*$ the i.saturated TBox
	obtained by exhaustive application of the classification rules.

	The canonical interpretation $\mathcal{I}_{\mathcal{T}^*}$ induced by $\mathcal{T}^*$ is defined as follows:
	\begin{align*}
		\Delta^{\mathcal{I}_{\mathcal{T}^*}} =& \left\{ M \mid M \text{ is a set of concept names in $\func{sig}(\mathcal{T})$ occurring in $\mathcal{T}^*$} \right\},\\
		A^{\mathcal{I}_{\mathcal{T}^*}} =& \left\{ M \in \Delta^{\mathcal{I}_{\mathcal{T}^*}} \mid M \sqsubseteq \left\{ A \right\} \in \mathcal{T}^* \right\},\\
		\begin{split}
		s^{\mathcal{I}_{\mathcal{T}^*}} =& \left\{ (M,N) \in \Delta^{\mathcal{I}_{\mathcal{T}^*}} \times \Delta^{\mathcal{I}_{\mathcal{T}^*}} \mid M \sqsubseteq \exists s.N \in \mathcal{T}^* \text{ and $N$ is maximal,} \right. \\
										&\quad\left. \text{i.e.\ there is no } N' \supsetneq N \text{ such that } M \sqsubseteq \exists s.N' \in \mathcal{T}^* \right\} \cup
		\end{split}\\
		\begin{split}
		& \left\{ (N,M) \in \Delta^{\mathcal{I}_{\mathcal{T}^*}} \times \Delta^{\mathcal{I}_{\mathcal{T}^*}} \mid M \sqsubseteq \exists s^-.N \in \mathcal{T}^* \text{ and $N$ is maximal,} \right. \\
										&\quad\left. \text{i.e.\ there is no } N' \supsetneq N \text{ such that } M \sqsubseteq \exists s^-.N' \in \mathcal{T}^* \right\}
		\end{split}
	\end{align*}
	where $A$ is a concept name in $\func{sig}(\mathcal{T})$, and $s$ a role name in $\func{sig}(\mathcal{T})$.
\end{definition}

\begin{lemma}
	Let $r$ be a role name or the inverse of a role name.
	Then
	\begin{align*}
		\begin{split}
			r^{\mathcal{I}_{\mathcal{T}^*}} =& \left\{ (M,N) \in \Delta^{\mathcal{I}_{\mathcal{T}^*}} \times \Delta^{\mathcal{I}_{\mathcal{T}^*}} \mid M \sqsubseteq \exists r.N \in \mathcal{T}^*, N \text{ is maximal}\right\}\cup\\
											 &\left\{ (N,M) \in \Delta^{\mathcal{I}_{\mathcal{T}^*}} \times \Delta^{\mathcal{I}_{\mathcal{T}^*}} \mid M \sqsubseteq \exists r^-.N \in \mathcal{T}^*, N \text{ is maximal}\right\}
		\end{split}
	\end{align*}
\end{lemma}
\begin{proof}
	If $r = s$ is a role name, then the identity holds by the definition of $s^{\mathcal{I}_{\mathcal{T}^*}}$.
	Otherwise: $r = s^-$ for a role name $s$.
	Then the identity follows from the fact that $r^- = s$,
	the semantics of the inverse operator ${}^-$,
	and the definition $s^{\mathcal{I}_{\mathcal{T}^*}}$.
\end{proof}
\begin{lemma}
	The canonical interpretation induced by the i.saturated TBox $\mathcal{T}^*$ is a model of $\mathcal{T}^*$.
\end{lemma}
\begin{proof}
	All the GCIs in $\mathcal{T}^*$ are $\mathcal{T}$-i.sequents.
	We consider the following cases (see Definition 6.16):
	\begin{itemize}
		\item $K \sqsubseteq \left\{ A \right\} \in \mathcal{T}^*$:
			let $M \in K^{\mathcal{I}_{\mathcal{T}^*}}$, i.e.\ $M \sqsubseteq \left\{ B \right\} \in \mathcal{T}^*$ for all $B \in K$.
			Then rule i.CR2 yields $M \sqsubseteq \left\{ A \right\} \in \mathcal{T}^*$,
			and thus $M \in A^{\mathcal{I}_{\mathcal{T}^*}}$.
		\item $K \sqsubseteq \exists r.K' \in \mathcal{T}^*$:
			assume that $M \in K^{\mathcal{I}_{\mathcal{T}^*}}$, i.e.\ $M \sqsubseteq \left\{ B \right\} \in \mathcal{T}^*$ for all $B \in K$.
			Then i.CR2 yields $M \sqsubseteq \exists r.K' \in \mathcal{T}^*$,
			and thus there is a maximal set $K'' \supseteq K'$ with $M \sqsubseteq \exists r.K'' \in \mathcal{T}^*$
			and $(M, K'') \in r^{\mathcal{I}_{\mathcal{T}^*}}$.
			Since $K''$ occurs in $\mathcal{T}^*$,
			i.CR1 yields $K '' \sqsubseteq \left\{ A \right\} \in \mathcal{T}^*$ for all $A \in K''$.
			Since $K' \subseteq K''$, this implies $K'' \in K'^{\mathcal{I}_{\mathcal{T}^*}}$.
			This shows $M \in \left( \exists r.K' \right)^{\mathcal{I}_{\mathcal{T}^*}}$.
		\item $K \sqsubseteq \forall r.A \in \mathcal{T}^*$:
			Assume that $M_1 \in K^{\mathcal{I}_{\mathcal{T}^*}}$ and there is $M_2$ such that
			$\left( M_1, M_2  \right) \in r^{\mathcal{I}_{\mathcal{T}^*}}$.
			We must show $M_2 \in A^{\mathcal{I}_{\mathcal{T}^*}}$.
			By the definition of $\mathcal{I}_{\mathcal{T}^*}$,
			$M_1 \in K^{\mathcal{I}_{\mathcal{T}^*}}$ yields
			$M_1 \sqsubseteq \left\{ B \right\} \in \mathcal{T}^*$ for all $B \in K$.
			Because of i.CR2, $M_1 \sqsubseteq \forall r.\left\{ A \right\} in \mathcal{T}^*$.
			There are two reasons, why $\left( M_1, M_2 \right)$ belongs to $r^{\mathcal{I}_{\mathcal{T}^*}}$:
			\begin{itemize}
				\item First, assume that $M_1 \sqsubseteq \exists r. M_2 \in \mathcal{T}^*$
					where $M_2$ is maximal with this property.
					Then, i.CR4 yields $M_1 \sqsubseteq \exists r.\left( M_2 \cup \left\{ A \right\} \right) \in \mathcal{T}^*$.
					Maximality of $M_2$ yieds $A \in M_2$.
					Since $M_2$ occurs in $\mathcal{T}^*$, i.CR1 yields $M_2 \sqsubseteq \left\{ A \right\} \in \mathcal{T}^*$,
					and thus $M_2 \in A^{\mathcal{I}_{\mathcal{T}^*}}$ as required.
				\item Second, assume that $M_2 \sqsubseteq \exists r^-.M_1 \in \mathcal{T}^*$, where $M_1$ is maximal.
					Then i.CR3 yields $M_2 \sqsubseteq \left\{ A \right\} \in \mathcal{T}^*$,
					and thus $M_2 \in A^{\mathcal{I}_{\mathcal{T}^*}}$.
					\qedhere
			\end{itemize}
	\end{itemize}
\end{proof}
