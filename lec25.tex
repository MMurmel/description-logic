\subsection{data complexity of OMQA in $\mathcal{EL}$}
\medskip
\lecture{25}{20.07.2021}{}
\begin{prop}
	In $\mathcal{EL}$, the query entailment problem for conjunctive queries is \textsc{P}-hard w.r.t.\ data complexity.
\end{prop}
\begin{proof}
	By \textsc{LogSpace}-reduction of path system accessibility.

	A path system is of the form $P = (N, E, S, t)$ where
	\begin{itemize}
		\item $N$ is a finite set of nodes,
		\item $E \sqsubseteq N \times N \times N$ is an accessibility relation
			(its elements are called edges),
		\item $S \subseteq N$ is a set of source nodes,
		\item and $t \in N$ is a terminal node.
	\end{itemize}
	The set of accessible nodes of $P$ is the smallest set of nodes such that
	\begin{itemize}
		\item every element of $S$ is accessible,
		\item if $n_1, n_2$ are accessible and $(n, n_1, n_2) \in E$, then $n$ is accessible.
	\end{itemize}
	The path accessibility problem is the question if for a given path system $P = (N, E, S, t)$
	the terminal node $t$ is accessible.
	It is known to be \textsc{P}-complete.

	We provide the following reduction:
	\begin{align*}
		\mathcal{T} &= \left\{ \exists P_1.A \sqsubseteq B_1,\quad \exists P_2.A \sqsubseteq B_2,\quad B_1 \sqcap B_2 \sqsubseteq A,\quad \exists P_3.A \sqsubseteq A \right\}\\
		q &= A(x)\\
		\mathcal{A} &= \left\{ n:A \mid n \in S \right\} \cup \\
					&\quad\,\left\{(e,j):P_1, \quad(e,k):P_2, \quad(n,e):P_3 \mid e = (n,j,k) \in E \right\}
	\end{align*}
	We then get
	\[
		(\mathcal{T}, \mathcal{A}) \vDash A(t) \iff t \text{ is accessible in } P
	.\]

	"$\implies$": By its contrapositive.
	Assume that $t$ is not accessible in P.
	We construct a model $\mathcal{I}$ of $\mathcal{K} = (\mathcal{T}, \mathcal{A})$
	such that $t^{\mathcal{I}} \notin A^{\mathcal{I}}$.
	We define $\mathcal{I}$ as follows:
	\begin{align*}
		&\Delta^{\mathcal{I}} \coloneqq N \cup E\\
		&e^{\mathcal{I}} \coloneqq e\\
		&n^{\mathcal{I}} \coloneqq n\\
		&P_1^{\mathcal{I}}, P_2^{\mathcal{I}}, P_3^{\mathcal{I}} \text{consist exactly of the tuples required by $\mathcal{A}$}\\
		&B_1^{\mathcal{I}} \coloneqq \left\{ e = (n,j,k) \mid j \text{ is accessible in } P \right\}\\
		&B_2^{\mathcal{I}} \coloneqq \left\{ e = (n,j,k) \mid k \text{ is accessible in } P \right\}\\
		&A^{\mathcal{I}} \coloneqq \left\{ n \in N \mid n \text{ is accessible in } P \right\} \cup\\
		&\;\;\,\qquad \left\{ e = (n,j,k) \mid j \text{ and } k \text{ are accessible in } P \right\}\\
	\end{align*}
	for $n \in N$ and $e \in E$.
	It is easy to see that $\mathcal{I}$ is a model of $\mathcal{K}$.
	Since $t$ is not accessible, we have $t^{\mathcal{I}} = t \notin A^{\mathcal{I}}$.

	"$\impliedby$":
	Assume $t$ is accessible in $P$.
	We prove by induction on the length of the derivation that yields accessibility that
	\[
		\text{node $n$ is accessible} \implies \mathcal{K} \vDash A(n)
	.\]
	\underline{base case:} $n \in S$.
	Then $A(n) \in \mathcal{A}$ and thus $\mathcal{K} \vDash A(n)$.
	\newline
	\underline{induction step}: There is an edge $e=(n, j, k)$ such that $j$ and $k$ are accessible.
	By induction we can assume that $\mathcal{K} \vDash A(j)$ and $\mathcal{K} \vDash A(k)$.
	Thus, in every model $\mathcal{I}$ of $\mathcal{K}$ we have $j^{\mathcal{I}} \in A^{\mathcal{I}}$ and $k^\mathcal{I} \in A^{\mathcal{I}}$.
	In addition, $\left( e^{\mathcal{I}}, j^{\mathcal{I}} \right) \in P_1^\mathcal{I}$ 
	and $\left( e^{\mathcal{I}}, k^{\mathcal{I}} \right) \in P_2^\mathcal{I}$.
	By the first three GCIs in $\mathcal{T}$ this yields $e^{\mathcal{I}} \in A^\mathcal{I}$.
	Since we also have $\left( n^{\mathcal{I}}, e^{\mathcal{I}} \right) \in P_3^\mathcal{I}$,
	the final GCI yields $n^{\mathcal{I}} \in A^{\mathcal{I}}$.
	This shows that $\mathcal{K} \vDash A(n)$.

	Sine $t$ is accessible in $P$, this yields $\mathcal{K} \vDash A(t)$.
\end{proof}

\subsection{OMQA using relational DB technology}
While complexity results in \textsc{P} are nice for OMQA, they can still be to slow for very large ABoxes, i.e.\ data.
In order to further optimize data complexity, we consider DLs for which the computation of certain answers
can be reduced to answering FO queries using a relational database system.

The general idea here is to reformulate the TBox and the conjunctive query into one FO query.
Since in data complexity their size is assumed to be constant, the resulting FO query is also constant in size.
The ABox is not changed, however it is viewed as a relational database, changing the open world approach of ABoxes
into the closed world approach of Databases.
A description logic for which this in possible is called \textit{FO-reducable}.

The DL-Lite family was created with this very purpose in mind.
It consists of several members, of which we will first look at $\text{DL-Lite}_{\text{core}}$.

\begin{mdframed}[frametitle= Syntax of $\text{DL-Lite}_{\text{core}}$]
	$\text{DL-Lite}_{\text{core}}$ has not only concept names and general concepts, but also an intermediate concept type,
	called basic concepts.
	Let $A$ be a concept name, then a basic concept is of one of the following forms:
	\[
		A, \quad \exists r.\top \quad\text{or}\quad \exists r^-.\top
	.\]

	General concepts are basic concepts or the negation of a basic concept.

	GCIs are of the form $B \sqsubseteq C$, where $B$ is a basic concept and $C$ is a general concept.

	The assertion of an ABox can be of the form $a:A$ or $(a,b):r$, where $a,b \in \mathscr{I}$, $A$ is a concept name and $r$ is a role name.
\end{mdframed}

The reformulating of the CQ and the TBox is not done formally in the lecture, but by example (see lecture slides).

$\text{DL-Lite}_{\text{core}}$ is extended by $\text{DL-Lite}_{\mathcal{R}}$ (additional role inclusions $r_1 \sqsubseteq r_2$ and $r_1 \sqsubseteq \neg r_2$)
and $\text{DL-Lite}_{\mathcal{F}}$ (additional functionality axioms: $\top \sqsubseteq ( \leq 1r)$ and $\top \sqsubseteq ( \leq 1r^-)$).
All three of them are FO-reducable.
FO-reducability implies data complexity in $\textsc{AC}^0$ for query answering.
